\begin{frame}
\frametitle{Tipo de soluciones}

Heurística:
\begin{block}{Definición}
Las heurísticas, al igual que los algoritmos, también son una secuencia de pasos bien definida con la diferencia de que dada una
entrada $I$, produce una salida $o \in \texttt{OUTPUT}$ con $\texttt{OUTPUT}$ un
conjunto de posibles valores de solución. Durante la ejecución de una heurística, la condición de término es más ambigua.
\end{block}

\pause

Ejemplos: 


\begin{itemize}
\item Heurísticas usadas para optimización de redes neuronales.
\item Búsqueda tabú.
\item Algoritmos (heurísticos) genéticos.
\end{itemize}

\pause

\begin{itemize}
		\item[$\blacksquare$] Ventaja: Reducción significativa en tiempo de respuesta.
		\item[$\blacksquare$] Desventaja: Desempeño pobre en el peor de los casos.
\end{itemize}

\end{frame}