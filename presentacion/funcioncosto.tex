\begin{frame}
\frametitle{Funciones de las abejas}
\begin{itemize}
\item Abeja exploradora.
\begin{itemize}
\item \texttt{\_busca\_fuente}: Copia la mejor fuente reportada en la colmena y genera un nuevo tablero $B_{n}$ a partir de una trayectoria $\sigma_{n-1}$. $B_{n}$ es ahora su fuente asociada.
\end{itemize}
\pause

\item Abeja trabajadora.
\begin{itemize}
\item \texttt{\_explotar}: Genera un $\sigma_{n}$ sobre su tablero asociado y lo aplica para obtener un tablero nuevo $B_{n+1}$.
\end{itemize}
\pause

\item Abeja observadora.
\begin{itemize}
\item \texttt{\_observadoras}: 
\begin{enumerate}
 \item Se le asigna una fuente de alguna abeja trabajadora. Todas las abejas observadoras tienen que poseer una fuente asociada.
 \item Corta parcialmente la ultima trayectoria $\sigma$ y la completa hasta encontrar una fuente ``cercana''.
 \item Si la fuente cercana es \textit{mejor} a la fuente original, avisa a la colmena. 
\end{enumerate}
\end{itemize}
\pause
\item Todas las abejas.
\begin{itemize}
\item \texttt{\_nectar}: Función de costo o función \textit{fitness}.
\end{itemize}
\end{itemize}
\end{frame}


\begin{frame}
\frametitle{Función de costo}
\begin{itemize}
\item ¿Cuál es el objetivo de la función de costo?
\pause
\begin{itemize}
\item Primer objetivo: que la heurística elimine $n \geq 10$ filas.
\pause
\item Objetivo principal: eliminar $n \geq 20$ filas.
\end{itemize}
\pause
\item ¿Cómo se logra orientar a la heurística hacia el objetivo? 
\pause
Suponer la siguiente entrada: \texttt{L = \{I, Rg, Lg, Sq, 
I, T, Ls, Rs, I\}}.
\begin{columns}
\column{0.45\textwidth}
\begin{figure}
\scalebox{.40}{\begin{figure}[H]
\centering
% 4 es Sq
% 1 es I
% 2 es L(Lg) inversa y 6 es L(Rg)
% 3 es S(Rs) y Z es 7(Ls)
% 5 es T 
% 1, 6, 2, 4, 1, 5, 7, 3, 1
% I, Rg, Lg, Sq, I, T, Ls, Rs, I
\begin{tikzpicture}[omino/nodes/.style={shape=rectangle, rounded corners, inner sep=+0pt, minimum size=1cm-2\pgflinewidth}]
	\path [omino={at={-2,0}, rotate=0}][tetris=1];
	\path [omino={at={4,0}, rotate=0}][tetris=1];
    \path [omino={at={2,0}, rotate=0}][tetris=4];
	\path [omino={at={1,0}, rotate=0}][tetris=2];	
	\path [omino={at={0,0}, rotate=0}][tetris=6];
	\path [omino={at={5,0}, rotate=0}][tetris=5];
	\path [omino={at={7,1}, rotate=0}][tetris=7];
	\path [omino={at={-2,4}, rotate=0}][tetris=3];
	\path [omino={at={0,3}, rotate=0}][tetris=1];
%    \path [omino/at=0:1] [tetris=2];
%    \path [omino/at=0:2] [tetris=3];
%    \path [omino={at=0:3, rotate=-90, x mirror}][tetris=5];
%    \path [omino={at={5,1}, rotate=-90}][tetris=3];
%    \path [omino={at={5,2}, rotate=-90}][tetris=2];
%    \path [omino={at={4,2}, x mirror}][tetris=5];
%    \path [omino={at={1,3}}][tetris=4];
\end{tikzpicture}
\caption{Una mala función de costo deja muchas casillas atrapadas y cubiertas.} \label{fig:mal-juego}
\end{figure}}
\end{figure}
\pause
\column{0.1\textwidth}
\begin{center}
$\Longrightarrow$
\end{center}
\column{0.45\textwidth}
\begin{figure}
\scalebox{.40}{\begin{tikzpicture}[omino/nodes/.style={shape=rectangle, rounded corners, inner sep=+0pt, minimum size=1cm-2\pgflinewidth}]
	\path [omino={at={-1,-2}, rotate=0}][tetris=1];
	\path [omino={at={3,1}, rotate=90}][tetris=1];
    \path [omino={at={1,-1}, rotate=0}] [tetris=4];
	\path [omino={at={3,0}, rotate=180}][tetris=2];	
	\path [omino={at={0,0}, rotate=180}][tetris=6];
	\path [omino={at={6,-2}, rotate=90}][tetris=5];
	\path [omino={at={4,-1}, rotate=0}][tetris=3];
	\path [omino={at={8,-2}, rotate=0}][tetris=1];
	\path [omino={at={7,-2}, rotate=0}] [tetris=7];	
%    \path [omino/at=0:1] [tetris=2];
%    \path [omino/at=0:2] [tetris=3];
%    \path [omino={at=0:3, rotate=-90, x mirror}][tetris=5];
%    \path [omino={at={5,1}, rotate=-90}][tetris=3];
%    \path [omino={at={5,2}, rotate=-90}][tetris=2];
%    \path [omino={at={4,2}, x mirror}][tetris=5];
%    \path [omino={at={1,3}}][tetris=4];
\end{tikzpicture}}
\end{figure}
\end{columns}

\item ¿Cómo se sabe que una función de costo funciona acorde a los requerimientos?
\pause
\textcolor{red}{Experimentación}.
\end{itemize}


\end{frame}

\begin{frame}
\frametitle{Filas entre pesos negativos}
\begin{displaymath}
   f = \frac{1 + \texttt{filas-removidas}}{\texttt{horizontalidad} + \texttt{atrapados} + \texttt{cubiertos} + \texttt{altura}} .
\end{displaymath}
\pause 
\begin{columns}
\column{0.45\textwidth}
\begin{figure}
\scalebox{.40}{\begin{figure}[H]
\centering
% 4 es Sq
% 1 es I
% 2 es L(Lg) inversa y 6 es L(Rg)
% 3 es S(Rs) y Z es 7(Ls)
% 5 es T 
% 1, 1, 4, 4, 4, 4
% I, I, Sq, Sq, Sq, Sq
\begin{tikzpicture}[omino/nodes/.style={shape=rectangle, rounded corners, inner sep=+0pt, minimum size=1cm-2\pgflinewidth}]
	\path [omino={at={0,0}, rotate=0}][tetris=2];
	\path [omino={at={4,0}, rotate=0}][tetris=4];
	\path [omino={at={5,2}, rotate=90}][tetris=1];
	\path [omino={at={9,0}, rotate=90}][tetris=1];	
	\path [omino={at={6,1}, rotate=0}][tetris=4];
	\path [omino={at={9,3}, rotate=180}][tetris=5];
\end{tikzpicture}
\caption{De izquierda a derecha las fichas son \texttt{Lg, I1, Sq1, Sq2, I2, T}.} \label{fig:eval-tablero}
\end{figure}}
\end{figure}
\begin{itemize}
\pause
\item $\texttt{filas-removidas} = 0$
\item $\texttt{horizontalidad} = 1$
\item $\texttt{atrapados} = 1$
\item $\texttt{cubiertos} = 7$
\item $\texttt{altura} = 4$ 
\end{itemize}
\begin{displaymath}
    \frac{1 + \texttt{0}}{\texttt{1} + \texttt{1} + \texttt{7} + \texttt{4}} = \frac{1}{13} \approx 0.077.
\end{displaymath}
\pause
\column{0.1\textwidth}
\begin{center}
VS.
\end{center}
\column{0.45\textwidth}
\begin{figure}
\scalebox{.40}{\begin{figure}[H]
\centering
% 4 es Sq
% 1 es I
% 2 es L(Lg) inversa y 6 es L(Rg)
% 3 es S(Rs) y Z es 7(Ls)
% 5 es T 
% 1, 1, 4, 4, 4, 4
% I, I, Sq, Sq, Sq, Sq
\begin{tikzpicture}[omino/nodes/.style={shape=rectangle, rounded corners, inner sep=+0pt, minimum size=1cm-2\pgflinewidth}]
	\path [omino={at={9,2}, rotate=180}][tetris=2];
	\path [omino={at={2,0}, rotate=0}][tetris=4];
	\path [omino={at={7,0}, rotate=90}][tetris=1];
	\path [omino={at={8,1}, rotate=90}][tetris=1];	
	\path [omino={at={0	,0}, rotate=0}][tetris=4];
	\path [omino={at={3,2}, rotate=-90}][tetris=5];
\end{tikzpicture}
\caption{Tablero que toma las piezas de la \cref{fig:eval-tablero} y la función de filas entre pesos negativos.} \label{fig:eval-tablero-dos}
\end{figure}}
\end{figure}
\begin{itemize}
\pause
\item $\texttt{filas-removidas} = 2$
\item $\texttt{horizontalidad} = 1$
\item $\texttt{atrapados} = 0$
\item $\texttt{cubiertos} = 0$
\item $\texttt{altura} = 4$ 
\end{itemize}
\begin{displaymath}
    \frac{1 + (2 \times c)}{1 + 0 + 0 + 1} = \frac{1 + 4}{2} = \frac{5}{2} = 2.5.
\end{displaymath}
\end{columns}

\end{frame}

\begin{frame}
\frametitle{Resultados filas entre pesos negativos}
En una colmena de 100 abejas, los resultados fueron los siguientes:
\begin{figure}
\scalebox{.7}{% GNUPLOT: LaTeX picture
\setlength{\unitlength}{0.240900pt}
\ifx\plotpoint\undefined\newsavebox{\plotpoint}\fi
\sbox{\plotpoint}{\rule[-0.200pt]{0.400pt}{0.400pt}}%
\begin{picture}(1500,900)(0,0)
\sbox{\plotpoint}{\rule[-0.200pt]{0.400pt}{0.400pt}}%
\put(131.0,131.0){\rule[-0.200pt]{4.818pt}{0.400pt}}
\put(111,131){\makebox(0,0)[r]{$0$}}
\put(1419.0,131.0){\rule[-0.200pt]{4.818pt}{0.400pt}}
\put(131.0,313.0){\rule[-0.200pt]{4.818pt}{0.400pt}}
\put(111,313){\makebox(0,0)[r]{$5$}}
\put(1419.0,313.0){\rule[-0.200pt]{4.818pt}{0.400pt}}
\put(131.0,495.0){\rule[-0.200pt]{4.818pt}{0.400pt}}
\put(111,495){\makebox(0,0)[r]{$10$}}
\put(1419.0,495.0){\rule[-0.200pt]{4.818pt}{0.400pt}}
\put(131.0,677.0){\rule[-0.200pt]{4.818pt}{0.400pt}}
\put(111,677){\makebox(0,0)[r]{$15$}}
\put(1419.0,677.0){\rule[-0.200pt]{4.818pt}{0.400pt}}
\put(131.0,859.0){\rule[-0.200pt]{4.818pt}{0.400pt}}
\put(111,859){\makebox(0,0)[r]{$20$}}
\put(1419.0,859.0){\rule[-0.200pt]{4.818pt}{0.400pt}}
\put(131.0,131.0){\rule[-0.200pt]{0.400pt}{4.818pt}}
\put(131,90){\makebox(0,0){$0$}}
\put(131.0,839.0){\rule[-0.200pt]{0.400pt}{4.818pt}}
\put(393.0,131.0){\rule[-0.200pt]{0.400pt}{4.818pt}}
\put(393,90){\makebox(0,0){$200$}}
\put(393.0,839.0){\rule[-0.200pt]{0.400pt}{4.818pt}}
\put(654.0,131.0){\rule[-0.200pt]{0.400pt}{4.818pt}}
\put(654,90){\makebox(0,0){$400$}}
\put(654.0,839.0){\rule[-0.200pt]{0.400pt}{4.818pt}}
\put(916.0,131.0){\rule[-0.200pt]{0.400pt}{4.818pt}}
\put(916,90){\makebox(0,0){$600$}}
\put(916.0,839.0){\rule[-0.200pt]{0.400pt}{4.818pt}}
\put(1177.0,131.0){\rule[-0.200pt]{0.400pt}{4.818pt}}
\put(1177,90){\makebox(0,0){$800$}}
\put(1177.0,839.0){\rule[-0.200pt]{0.400pt}{4.818pt}}
\put(1439.0,131.0){\rule[-0.200pt]{0.400pt}{4.818pt}}
\put(1439,90){\makebox(0,0){$1000$}}
\put(1439.0,839.0){\rule[-0.200pt]{0.400pt}{4.818pt}}
\put(131.0,131.0){\rule[-0.200pt]{0.400pt}{175.375pt}}
\put(131.0,131.0){\rule[-0.200pt]{315.097pt}{0.400pt}}
\put(1439.0,131.0){\rule[-0.200pt]{0.400pt}{175.375pt}}
\put(131.0,859.0){\rule[-0.200pt]{315.097pt}{0.400pt}}
\put(30,495){\rotatebox{-270}{\makebox(0,0){Filas eliminadas}}
}\put(785,29){\makebox(0,0){Semillas probadas}}
\put(1279,818){\makebox(0,0)[r]{Filas entre pesos neg}}
\put(1299.0,818.0){\rule[-0.200pt]{24.090pt}{0.400pt}}
\put(132,313){\usebox{\plotpoint}}
\multiput(132.58,313.00)(0.492,1.530){21}{\rule{0.119pt}{1.300pt}}
\multiput(131.17,313.00)(12.000,33.302){2}{\rule{0.400pt}{0.650pt}}
\multiput(144.00,349.58)(0.809,0.499){143}{\rule{0.747pt}{0.120pt}}
\multiput(144.00,348.17)(116.450,73.000){2}{\rule{0.373pt}{0.400pt}}
\multiput(262.00,422.58)(0.899,0.499){143}{\rule{0.818pt}{0.120pt}}
\multiput(262.00,421.17)(129.303,73.000){2}{\rule{0.409pt}{0.400pt}}
\multiput(393.00,495.58)(1.078,0.500){361}{\rule{0.962pt}{0.120pt}}
\multiput(393.00,494.17)(390.004,182.000){2}{\rule{0.481pt}{0.400pt}}
\multiput(785.00,677.58)(3.008,0.499){215}{\rule{2.500pt}{0.120pt}}
\multiput(785.00,676.17)(648.811,109.000){2}{\rule{1.250pt}{0.400pt}}
\put(131.0,131.0){\rule[-0.200pt]{0.400pt}{175.375pt}}
\put(131.0,131.0){\rule[-0.200pt]{315.097pt}{0.400pt}}
\put(1439.0,131.0){\rule[-0.200pt]{0.400pt}{175.375pt}}
\put(131.0,859.0){\rule[-0.200pt]{315.097pt}{0.400pt}}
\end{picture}
}
\pause
\begin{itemize}
\item Se obtuvieron como mejor solución 18 filas removidas.
\end{itemize}
\end{figure}
\end{frame}


\begin{frame}
\frametitle{\textit{Raining skyline}}
\begin{itemize}
\item Optimización de casillas disponibles. ¿En qué consiste?
\end{itemize}
\begin{columns}
\column{0.45\textwidth}
\begin{figure}
\scalebox{.40}{% 4 es Sq
% 1 es I
% 2 es L(Lg) inversa y 6 es L(Rg)
% 3 es S(Rs) y Z es 7(Ls)
% 5 es T 
% 1, 1, 4, 4, 4, 4
% I, I, Sq, Sq, Sq, Sq
\begin{tikzpicture}[omino/nodes/.style={shape=rectangle, rounded corners, inner sep=+0pt, minimum size=1cm-2\pgflinewidth}]
	
	\path [omino={at={-2,0}, rotate=0}][tetris=0];
	\path [omino={at={-1,0}, rotate=0}][tetris=0];
	\path [omino={at={0,0}, rotate=0}][tetris=0];
	\path [omino={at={1,0}, rotate=0}][tetris=0];
	\path [omino={at={2,0}, rotate=0}][tetris=0];
	\path [omino={at={3,0}, rotate=0}][tetris=0];
	\path [omino={at={4,0}, rotate=0}][tetris=0];
	\path [omino={at={5,0}, rotate=0}][tetris=0];
	\path [omino={at={6,0}, rotate=0}][tetris=0];
	\path [omino={at={7,0}, rotate=0}][tetris=0];
	
	\path [omino={at={-2,1}, rotate=0}][tetris=0];
	\path [omino={at={-1,1}, rotate=0}][tetris=0];
	\path [omino={at={0,1}, rotate=0}][tetris=0];	
	\path [omino={at={1,1}, rotate=0}][tetris=0];
	\path [omino={at={2,1}, rotate=0}][tetris=0];
	\path [omino={at={3,1}, rotate=0}][tetris=0];
	\path [omino={at={4,1}, rotate=0}][tetris=0];
	\path [omino={at={5,1}, rotate=0}][tetris=0];
	\path [omino={at={6,1}, rotate=0}][tetris=0];
	\path [omino={at={7,1}, rotate=0}][tetris=0];

	
	\path [omino={at={-2,2}, rotate=0}][tetris=0];
	\path [omino={at={-1,2}, rotate=0}][tetris=0];
	\path [omino={at={0,2}, rotate=0}][tetris=0];
	\path [omino={at={1,2}, rotate=0}][tetris=0];
	\path [omino={at={2,2}, rotate=0}][tetris=0];
	\path [omino={at={3,2}, rotate=0}][tetris=0];
	\path [omino={at={4,2}, rotate=0}][tetris=0];
	\path [omino={at={5,2}, rotate=0}][tetris=0];
	\path [omino={at={6,2}, rotate=0}][tetris=0];
	\path [omino={at={7,2}, rotate=0}][tetris=0];
	
	\path [omino={at={-2,3}, rotate=0}][tetris=0];
	\path [omino={at={-1,3}, rotate=0}][tetris=0];
	\path [omino={at={0,3}, rotate=0}][tetris=0];
	\path [omino={at={1,3}, rotate=0}][tetris=0];
	\path [omino={at={2,3}, rotate=0}][tetris=0];
	\path [omino={at={3,3}, rotate=0}][tetris=0];
	\path [omino={at={4,3}, rotate=0}][tetris=0];
	\path [omino={at={5,3}, rotate=0}][tetris=0];
	\path [omino={at={6,3}, rotate=0}][tetris=0];
	\path [omino={at={7,3}, rotate=0}][tetris=0];	
	
	\path [omino={at={-2,4}, rotate=0}][tetris=0];
	\path [omino={at={-1,4}, rotate=0}][tetris=0];
	\path [omino={at={0,4}, rotate=0}][tetris=0];
	\path [omino={at={1,4}, rotate=0}][tetris=0];
	\path [omino={at={2,4}, rotate=0}][tetris=0];
	\path [omino={at={3,4}, rotate=0}][tetris=0];
	\path [omino={at={4,4}, rotate=0}][tetris=0];
	\path [omino={at={5,4}, rotate=0}][tetris=0];
	\path [omino={at={6,4}, rotate=0}][tetris=0];
	\path [omino={at={7,4}, rotate=0}][tetris=0];	
	
	\path [omino={at={-2,5}, rotate=0}][tetris=0];
	\path [omino={at={-1,5}, rotate=0}][tetris=0];
	\path [omino={at={0,5}, rotate=0}][tetris=0];
	\path [omino={at={1,5}, rotate=0}][tetris=0];
	\path [omino={at={2,5}, rotate=0}][tetris=0];
	\path [omino={at={3,5}, rotate=0}][tetris=0];
	\path [omino={at={4,5}, rotate=0}][tetris=0];
	\path [omino={at={5,5}, rotate=0}][tetris=0];
	\path [omino={at={6,5}, rotate=0}][tetris=0];
	\path [omino={at={7,5}, rotate=0}][tetris=0];	
	
	\path [omino={at={-2,6}, rotate=0}][tetris=0];
	\path [omino={at={-1,6}, rotate=0}][tetris=0];
	\path [omino={at={0,6}, rotate=0}][tetris=0];
	\path [omino={at={1,6}, rotate=0}][tetris=0];
	\path [omino={at={2,6}, rotate=0}][tetris=0];
	\path [omino={at={3,6}, rotate=0}][tetris=0];
	\path [omino={at={4,6}, rotate=0}][tetris=0];
	\path [omino={at={5,6}, rotate=0}][tetris=0];
	\path [omino={at={6,6}, rotate=0}][tetris=0];
	\path [omino={at={7,6}, rotate=0}][tetris=0];			
	
	\path [omino={at={0,1}, rotate=90}][tetris=7];
	\path [omino={at={2,3}, rotate=90}][tetris=7];
	\path [omino={at={3,0}, rotate=0}][tetris=4];
	\path [omino={at={4,2}, rotate=0}][tetris=4];
	\path [omino={at={7,0}, rotate=90}][tetris=5];	
\end{tikzpicture}
}
\end{figure}
\pause
\column{0.1\textwidth}
\begin{center}
$\Longrightarrow$
\end{center}
\column{0.45\textwidth}
\begin{figure}
\scalebox{.40}{\input{./images/cierre-convexo.tex}}
\end{figure}
\end{columns}
\pause
\vspace{10mm}
\begin{displaymath}
  rs = \sum_{i=1,j=1}^{ancho, alto} j \enskip | \enskip (i,j) \in C_{sup}.
\end{displaymath} 
\hspace{10mm}
\begin{displaymath}
  f(T) = \frac{n - rs}{n}.
\end{displaymath} 
\end{frame}

\begin{frame}
\frametitle{Resultados de \textit{Raining skyline} sin ponderar}
\begin{figure}
\scalebox{.7}{% GNUPLOT: LaTeX picture
\setlength{\unitlength}{0.240900pt}
\ifx\plotpoint\undefined\newsavebox{\plotpoint}\fi
\sbox{\plotpoint}{\rule[-0.200pt]{0.400pt}{0.400pt}}%
\begin{picture}(1500,900)(0,0)
\sbox{\plotpoint}{\rule[-0.200pt]{0.400pt}{0.400pt}}%
\put(111.0,252.0){\rule[-0.200pt]{4.818pt}{0.400pt}}
\put(91,252){\makebox(0,0)[r]{$0$}}
\put(1419.0,252.0){\rule[-0.200pt]{4.818pt}{0.400pt}}
\put(111.0,859.0){\rule[-0.200pt]{4.818pt}{0.400pt}}
\put(91,859){\makebox(0,0)[r]{$5$}}
\put(1419.0,859.0){\rule[-0.200pt]{4.818pt}{0.400pt}}
\put(111.0,131.0){\rule[-0.200pt]{0.400pt}{4.818pt}}
\put(111,90){\makebox(0,0){$0$}}
\put(111.0,839.0){\rule[-0.200pt]{0.400pt}{4.818pt}}
\put(377.0,131.0){\rule[-0.200pt]{0.400pt}{4.818pt}}
\put(377,90){\makebox(0,0){$1000$}}
\put(377.0,839.0){\rule[-0.200pt]{0.400pt}{4.818pt}}
\put(642.0,131.0){\rule[-0.200pt]{0.400pt}{4.818pt}}
\put(642,90){\makebox(0,0){$2000$}}
\put(642.0,839.0){\rule[-0.200pt]{0.400pt}{4.818pt}}
\put(908.0,131.0){\rule[-0.200pt]{0.400pt}{4.818pt}}
\put(908,90){\makebox(0,0){$3000$}}
\put(908.0,839.0){\rule[-0.200pt]{0.400pt}{4.818pt}}
\put(1173.0,131.0){\rule[-0.200pt]{0.400pt}{4.818pt}}
\put(1173,90){\makebox(0,0){$4000$}}
\put(1173.0,839.0){\rule[-0.200pt]{0.400pt}{4.818pt}}
\put(1439.0,131.0){\rule[-0.200pt]{0.400pt}{4.818pt}}
\put(1439,90){\makebox(0,0){$5000$}}
\put(1439.0,839.0){\rule[-0.200pt]{0.400pt}{4.818pt}}
\put(111.0,131.0){\rule[-0.200pt]{0.400pt}{175.375pt}}
\put(111.0,131.0){\rule[-0.200pt]{319.915pt}{0.400pt}}
\put(1439.0,131.0){\rule[-0.200pt]{0.400pt}{175.375pt}}
\put(111.0,859.0){\rule[-0.200pt]{319.915pt}{0.400pt}}
\put(30,495){\rotatebox{-270}{\makebox(0,0){Filas eliminadas}}
}\put(775,29){\makebox(0,0){Semillas probadas}}
\put(111,616){\usebox{\plotpoint}}
\put(110.67,495){\rule{0.400pt}{29.149pt}}
\multiput(110.17,555.50)(1.000,-60.500){2}{\rule{0.400pt}{14.574pt}}
\put(112.17,252){\rule{0.400pt}{48.700pt}}
\multiput(111.17,393.92)(2.000,-141.921){2}{\rule{0.400pt}{24.350pt}}
\multiput(128.00,252.58)(1.188,0.500){483}{\rule{1.050pt}{0.120pt}}
\multiput(128.00,251.17)(574.821,243.000){2}{\rule{0.525pt}{0.400pt}}
\multiput(705.00,495.58)(0.687,0.500){483}{\rule{0.650pt}{0.120pt}}
\multiput(705.00,494.17)(332.651,243.000){2}{\rule{0.325pt}{0.400pt}}
\put(114.0,252.0){\rule[-0.200pt]{3.373pt}{0.400pt}}
\put(111,616){\makebox(0,0){$+$}}
\put(112,495){\makebox(0,0){$+$}}
\put(114,252){\makebox(0,0){$+$}}
\put(123,252){\makebox(0,0){$+$}}
\put(128,252){\makebox(0,0){$+$}}
\put(705,495){\makebox(0,0){$+$}}
\put(1039,738){\makebox(0,0){$+$}}
\put(1439,738){\makebox(0,0){$+$}}
\put(1039.0,738.0){\rule[-0.200pt]{96.360pt}{0.400pt}}
\put(111.0,131.0){\rule[-0.200pt]{0.400pt}{175.375pt}}
\put(111.0,131.0){\rule[-0.200pt]{319.915pt}{0.400pt}}
\put(1439.0,131.0){\rule[-0.200pt]{0.400pt}{175.375pt}}
\put(111.0,859.0){\rule[-0.200pt]{319.915pt}{0.400pt}}
\end{picture}
}
\end{figure}
\pause
\begin{itemize}
\item ¿Por qué los resultados no son los esperados?
\item ¿Por qué puede decrecer el número de filas eliminadas en una ``mejor'' semilla?
\end{itemize}
\end{frame}

\begin{frame}
\frametitle{Ponderación de \textit{Raining skyline}}
\begin{itemize}
\item[1]
\end{itemize}
\begin{displaymath}
  rs = \sum_{i=1,j=1}^{ancho, alto} j \enskip | \enskip j > b \enskip \land \enskip (i,b) \in C_{sup} 
  \end{displaymath} 

  \begin{displaymath}
    f(T) = \frac{rs}{\sum\limits_{i=1}^{alto} (ancho \times i)}.
\end{displaymath} 
\pause
\begin{itemize}
\item[2]
\end{itemize}
 \begin{displaymath}
    f(T) = \left(\frac{rs}{\sum\limits_{i=1}^{alto} (ancho \times i)}\right) \times (1 + fe).
\end{displaymath} 
\end{frame}

\begin{frame}
\frametitle{Resultados de \textit{Raining skyline} ponderada}
En una colmena de 100 abejas, los resultados fueron los siguientes:
\begin{figure}
\scalebox{.85}{% GNUPLOT: LaTeX picture
\setlength{\unitlength}{0.240900pt}
\ifx\plotpoint\undefined\newsavebox{\plotpoint}\fi
\sbox{\plotpoint}{\rule[-0.200pt]{0.400pt}{0.400pt}}%
\begin{picture}(1500,900)(0,0)
\sbox{\plotpoint}{\rule[-0.200pt]{0.400pt}{0.400pt}}%
\put(131.0,170.0){\rule[-0.200pt]{4.818pt}{0.400pt}}
\put(111,170){\makebox(0,0)[r]{$0$}}
\put(1419.0,170.0){\rule[-0.200pt]{4.818pt}{0.400pt}}
\put(131.0,269.0){\rule[-0.200pt]{4.818pt}{0.400pt}}
\put(111,269){\makebox(0,0)[r]{$5$}}
\put(1419.0,269.0){\rule[-0.200pt]{4.818pt}{0.400pt}}
\put(131.0,367.0){\rule[-0.200pt]{4.818pt}{0.400pt}}
\put(111,367){\makebox(0,0)[r]{$10$}}
\put(1419.0,367.0){\rule[-0.200pt]{4.818pt}{0.400pt}}
\put(131.0,465.0){\rule[-0.200pt]{4.818pt}{0.400pt}}
\put(111,465){\makebox(0,0)[r]{$15$}}
\put(1419.0,465.0){\rule[-0.200pt]{4.818pt}{0.400pt}}
\put(131.0,564.0){\rule[-0.200pt]{4.818pt}{0.400pt}}
\put(111,564){\makebox(0,0)[r]{$20$}}
\put(1419.0,564.0){\rule[-0.200pt]{4.818pt}{0.400pt}}
\put(131.0,662.0){\rule[-0.200pt]{4.818pt}{0.400pt}}
\put(111,662){\makebox(0,0)[r]{$25$}}
\put(1419.0,662.0){\rule[-0.200pt]{4.818pt}{0.400pt}}
\put(131.0,761.0){\rule[-0.200pt]{4.818pt}{0.400pt}}
\put(111,761){\makebox(0,0)[r]{$30$}}
\put(1419.0,761.0){\rule[-0.200pt]{4.818pt}{0.400pt}}
\put(131.0,859.0){\rule[-0.200pt]{4.818pt}{0.400pt}}
\put(111,859){\makebox(0,0)[r]{$35$}}
\put(1419.0,859.0){\rule[-0.200pt]{4.818pt}{0.400pt}}
\put(131.0,131.0){\rule[-0.200pt]{0.400pt}{4.818pt}}
\put(131,90){\makebox(0,0){$0$}}
\put(131.0,839.0){\rule[-0.200pt]{0.400pt}{4.818pt}}
\put(393.0,131.0){\rule[-0.200pt]{0.400pt}{4.818pt}}
\put(393,90){\makebox(0,0){$1000$}}
\put(393.0,839.0){\rule[-0.200pt]{0.400pt}{4.818pt}}
\put(654.0,131.0){\rule[-0.200pt]{0.400pt}{4.818pt}}
\put(654,90){\makebox(0,0){$2000$}}
\put(654.0,839.0){\rule[-0.200pt]{0.400pt}{4.818pt}}
\put(916.0,131.0){\rule[-0.200pt]{0.400pt}{4.818pt}}
\put(916,90){\makebox(0,0){$3000$}}
\put(916.0,839.0){\rule[-0.200pt]{0.400pt}{4.818pt}}
\put(1177.0,131.0){\rule[-0.200pt]{0.400pt}{4.818pt}}
\put(1177,90){\makebox(0,0){$4000$}}
\put(1177.0,839.0){\rule[-0.200pt]{0.400pt}{4.818pt}}
\put(1439.0,131.0){\rule[-0.200pt]{0.400pt}{4.818pt}}
\put(1439,90){\makebox(0,0){$5000$}}
\put(1439.0,839.0){\rule[-0.200pt]{0.400pt}{4.818pt}}
\put(131.0,131.0){\rule[-0.200pt]{0.400pt}{175.375pt}}
\put(131.0,131.0){\rule[-0.200pt]{315.097pt}{0.400pt}}
\put(1439.0,131.0){\rule[-0.200pt]{0.400pt}{175.375pt}}
\put(131.0,859.0){\rule[-0.200pt]{315.097pt}{0.400pt}}
\put(785,525){\makebox(0,0)[l]{Objetivo}}
\put(131,564){\line(1,0){1308}}
\put(30,495){\rotatebox{-270}{\makebox(0,0){Filas eliminadas}}
}\put(785,29){\makebox(0,0){Semillas probadas}}
\put(1279,818){\makebox(0,0)[r]{Función raining-skyline}}
\put(1299.0,818.0){\rule[-0.200pt]{24.090pt}{0.400pt}}
\put(131,229){\usebox{\plotpoint}}
\put(130.67,210){\rule{0.400pt}{4.577pt}}
\multiput(130.17,219.50)(1.000,-9.500){2}{\rule{0.400pt}{2.289pt}}
\put(132.17,170){\rule{0.400pt}{8.100pt}}
\multiput(131.17,193.19)(2.000,-23.188){2}{\rule{0.400pt}{4.050pt}}
\multiput(148.00,170.58)(7.158,0.498){77}{\rule{5.780pt}{0.120pt}}
\multiput(148.00,169.17)(556.003,40.000){2}{\rule{2.890pt}{0.400pt}}
\multiput(716.00,210.58)(4.251,0.498){75}{\rule{3.474pt}{0.120pt}}
\multiput(716.00,209.17)(321.789,39.000){2}{\rule{1.737pt}{0.400pt}}
\put(134.0,170.0){\rule[-0.200pt]{3.373pt}{0.400pt}}
\put(1045.0,249.0){\rule[-0.200pt]{94.915pt}{0.400pt}}
\sbox{\plotpoint}{\rule[-0.500pt]{1.000pt}{1.000pt}}%
\sbox{\plotpoint}{\rule[-0.200pt]{0.400pt}{0.400pt}}%
\put(1279,777){\makebox(0,0)[r]{Función raining-skyline ponderada}}
\sbox{\plotpoint}{\rule[-0.500pt]{1.000pt}{1.000pt}}%
\multiput(1299,777)(20.756,0.000){5}{\usebox{\plotpoint}}
\put(1399,777){\usebox{\plotpoint}}
\put(131,269){\usebox{\plotpoint}}
\multiput(131,269)(0.352,-20.753){3}{\usebox{\plotpoint}}
\put(135.26,210.00){\usebox{\plotpoint}}
\multiput(143,210)(2.574,-20.595){2}{\usebox{\plotpoint}}
\multiput(148,170)(20.322,4.222){28}{\usebox{\plotpoint}}
\multiput(716,288)(19.537,7.007){17}{\usebox{\plotpoint}}
\multiput(1045,406)(20.527,3.074){19}{\usebox{\plotpoint}}
\put(1439,465){\usebox{\plotpoint}}
\put(131,269){\makebox(0,0){$\circ$}}
\put(132,210){\makebox(0,0){$\circ$}}
\put(134,210){\makebox(0,0){$\circ$}}
\put(143,210){\makebox(0,0){$\circ$}}
\put(148,170){\makebox(0,0){$\circ$}}
\put(716,288){\makebox(0,0){$\circ$}}
\put(1045,406){\makebox(0,0){$\circ$}}
\put(1439,465){\makebox(0,0){$\circ$}}
\put(1349,777){\makebox(0,0){$\circ$}}
\sbox{\plotpoint}{\rule[-0.400pt]{0.800pt}{0.800pt}}%
\sbox{\plotpoint}{\rule[-0.200pt]{0.400pt}{0.400pt}}%
\put(1279,736){\makebox(0,0)[r]{Función raining-skyline ponderada por filas eliminadas}}
\sbox{\plotpoint}{\rule[-0.400pt]{0.800pt}{0.800pt}}%
\put(1299.0,736.0){\rule[-0.400pt]{24.090pt}{0.800pt}}
\put(131,308){\usebox{\plotpoint}}
\put(129.84,308){\rule{0.800pt}{14.213pt}}
\multiput(129.34,308.00)(1.000,29.500){2}{\rule{0.800pt}{7.107pt}}
\put(131.34,367){\rule{0.800pt}{9.395pt}}
\multiput(130.34,367.00)(2.000,19.500){2}{\rule{0.800pt}{4.698pt}}
\multiput(135.40,406.00)(0.516,2.441){11}{\rule{0.124pt}{3.756pt}}
\multiput(132.34,406.00)(9.000,32.205){2}{\rule{0.800pt}{1.878pt}}
\multiput(144.38,446.00)(0.560,12.850){3}{\rule{0.135pt}{12.840pt}}
\multiput(141.34,446.00)(5.000,52.350){2}{\rule{0.800pt}{6.420pt}}
\multiput(148.00,526.41)(4.865,0.502){111}{\rule{7.902pt}{0.121pt}}
\multiput(148.00,523.34)(551.600,59.000){2}{\rule{3.951pt}{0.800pt}}
\multiput(716.00,585.41)(9.022,0.506){31}{\rule{14.053pt}{0.122pt}}
\multiput(716.00,582.34)(299.833,19.000){2}{\rule{7.026pt}{0.800pt}}
\multiput(1045.00,604.41)(10.242,0.505){33}{\rule{15.960pt}{0.122pt}}
\multiput(1045.00,601.34)(360.874,20.000){2}{\rule{7.980pt}{0.800pt}}
\put(131,308){\makebox(0,0){$\ast$}}
\put(132,367){\makebox(0,0){$\ast$}}
\put(134,406){\makebox(0,0){$\ast$}}
\put(143,446){\makebox(0,0){$\ast$}}
\put(148,525){\makebox(0,0){$\ast$}}
\put(716,584){\makebox(0,0){$\ast$}}
\put(1045,603){\makebox(0,0){$\ast$}}
\put(1439,623){\makebox(0,0){$\ast$}}
\put(1349,736){\makebox(0,0){$\ast$}}
\sbox{\plotpoint}{\rule[-0.200pt]{0.400pt}{0.400pt}}%
\put(131.0,131.0){\rule[-0.200pt]{0.400pt}{175.375pt}}
\put(131.0,131.0){\rule[-0.200pt]{315.097pt}{0.400pt}}
\put(1439.0,131.0){\rule[-0.200pt]{0.400pt}{175.375pt}}
\put(131.0,859.0){\rule[-0.200pt]{315.097pt}{0.400pt}}
\end{picture}
}
\end{figure}
\end{frame}

\begin{frame}
\frametitle{Observaciones adicionales}
\begin{itemize}
\pause
\item Tamaño de la colmena se mantuvo en 100 abejas.
\pause
\item Cada función fue probada con al menos 10,000 semillas pseudoaleatorias. 
\pause
\item Se consiguieron muestras de números mayores al objetivo (71 filas eliminadas).
\pause 
\item Los resultados de ambas funciones arrojan muestras muy similares.
\begin{figure}
\scalebox{.5}{% GNUPLOT: LaTeX picture
\setlength{\unitlength}{0.240900pt}
\ifx\plotpoint\undefined\newsavebox{\plotpoint}\fi
\sbox{\plotpoint}{\rule[-0.200pt]{0.400pt}{0.400pt}}%
\begin{picture}(1500,900)(0,0)
\sbox{\plotpoint}{\rule[-0.200pt]{0.400pt}{0.400pt}}%
\put(131.0,131.0){\rule[-0.200pt]{4.818pt}{0.400pt}}
\put(111,131){\makebox(0,0)[r]{$0$}}
\put(1419.0,131.0){\rule[-0.200pt]{4.818pt}{0.400pt}}
\put(131.0,235.0){\rule[-0.200pt]{4.818pt}{0.400pt}}
\put(111,235){\makebox(0,0)[r]{$5$}}
\put(1419.0,235.0){\rule[-0.200pt]{4.818pt}{0.400pt}}
\put(131.0,339.0){\rule[-0.200pt]{4.818pt}{0.400pt}}
\put(111,339){\makebox(0,0)[r]{$10$}}
\put(1419.0,339.0){\rule[-0.200pt]{4.818pt}{0.400pt}}
\put(131.0,443.0){\rule[-0.200pt]{4.818pt}{0.400pt}}
\put(111,443){\makebox(0,0)[r]{$15$}}
\put(1419.0,443.0){\rule[-0.200pt]{4.818pt}{0.400pt}}
\put(131.0,547.0){\rule[-0.200pt]{4.818pt}{0.400pt}}
\put(111,547){\makebox(0,0)[r]{$20$}}
\put(1419.0,547.0){\rule[-0.200pt]{4.818pt}{0.400pt}}
\put(131.0,651.0){\rule[-0.200pt]{4.818pt}{0.400pt}}
\put(111,651){\makebox(0,0)[r]{$25$}}
\put(1419.0,651.0){\rule[-0.200pt]{4.818pt}{0.400pt}}
\put(131.0,755.0){\rule[-0.200pt]{4.818pt}{0.400pt}}
\put(111,755){\makebox(0,0)[r]{$30$}}
\put(1419.0,755.0){\rule[-0.200pt]{4.818pt}{0.400pt}}
\put(131.0,859.0){\rule[-0.200pt]{4.818pt}{0.400pt}}
\put(111,859){\makebox(0,0)[r]{$35$}}
\put(1419.0,859.0){\rule[-0.200pt]{4.818pt}{0.400pt}}
\put(131.0,131.0){\rule[-0.200pt]{0.400pt}{4.818pt}}
\put(131,90){\makebox(0,0){$0$}}
\put(131.0,839.0){\rule[-0.200pt]{0.400pt}{4.818pt}}
\put(393.0,131.0){\rule[-0.200pt]{0.400pt}{4.818pt}}
\put(393,90){\makebox(0,0){$200$}}
\put(393.0,839.0){\rule[-0.200pt]{0.400pt}{4.818pt}}
\put(654.0,131.0){\rule[-0.200pt]{0.400pt}{4.818pt}}
\put(654,90){\makebox(0,0){$400$}}
\put(654.0,839.0){\rule[-0.200pt]{0.400pt}{4.818pt}}
\put(916.0,131.0){\rule[-0.200pt]{0.400pt}{4.818pt}}
\put(916,90){\makebox(0,0){$600$}}
\put(916.0,839.0){\rule[-0.200pt]{0.400pt}{4.818pt}}
\put(1177.0,131.0){\rule[-0.200pt]{0.400pt}{4.818pt}}
\put(1177,90){\makebox(0,0){$800$}}
\put(1177.0,839.0){\rule[-0.200pt]{0.400pt}{4.818pt}}
\put(1439.0,131.0){\rule[-0.200pt]{0.400pt}{4.818pt}}
\put(1439,90){\makebox(0,0){$1000$}}
\put(1439.0,839.0){\rule[-0.200pt]{0.400pt}{4.818pt}}
\put(131.0,131.0){\rule[-0.200pt]{0.400pt}{175.375pt}}
\put(131.0,131.0){\rule[-0.200pt]{315.097pt}{0.400pt}}
\put(1439.0,131.0){\rule[-0.200pt]{0.400pt}{175.375pt}}
\put(131.0,859.0){\rule[-0.200pt]{315.097pt}{0.400pt}}
\put(30,495){\rotatebox{90}{\makebox(0,0){Filas eliminadas}}
}\put(785,29){\makebox(0,0){Número de pruebas}}
\put(1279,818){\makebox(0,0)[r]{Función raining-skyline}}
\put(1299.0,818.0){\rule[-0.200pt]{24.090pt}{0.400pt}}
\put(131,235){\usebox{\plotpoint}}
\multiput(131.00,235.58)(0.524,0.500){621}{\rule{0.519pt}{0.120pt}}
\multiput(131.00,234.17)(325.922,312.000){2}{\rule{0.260pt}{0.400pt}}
\multiput(785.00,547.58)(1.976,0.499){163}{\rule{1.676pt}{0.120pt}}
\multiput(785.00,546.17)(323.522,83.000){2}{\rule{0.838pt}{0.400pt}}
\multiput(1112.00,630.58)(3.921,0.498){81}{\rule{3.214pt}{0.120pt}}
\multiput(1112.00,629.17)(320.329,42.000){2}{\rule{1.607pt}{0.400pt}}
\put(458.0,547.0){\rule[-0.200pt]{78.774pt}{0.400pt}}
\sbox{\plotpoint}{\rule[-0.600pt]{1.200pt}{1.200pt}}%
\sbox{\plotpoint}{\rule[-0.200pt]{0.400pt}{0.400pt}}%
\put(1279,777){\makebox(0,0)[r]{Función de pesos pos/neg}}
\sbox{\plotpoint}{\rule[-0.600pt]{1.200pt}{1.200pt}}%
\put(1299.0,777.0){\rule[-0.600pt]{24.090pt}{1.200pt}}
\put(131,256){\usebox{\plotpoint}}
\multiput(131.00,258.24)(0.984,0.500){322}{\rule{2.664pt}{0.120pt}}
\multiput(131.00,253.51)(321.471,166.000){2}{\rule{1.332pt}{1.200pt}}
\multiput(458.00,424.24)(0.873,0.500){364}{\rule{2.398pt}{0.120pt}}
\multiput(458.00,419.51)(322.022,187.000){2}{\rule{1.199pt}{1.200pt}}
\multiput(1112.00,611.24)(1.574,0.500){198}{\rule{4.073pt}{0.120pt}}
\multiput(1112.00,606.51)(318.546,104.000){2}{\rule{2.037pt}{1.200pt}}
\put(131,256){\makebox(0,0){$\blacksquare$}}
\put(458,422){\makebox(0,0){$\blacksquare$}}
\put(785,609){\makebox(0,0){$\blacksquare$}}
\put(1112,609){\makebox(0,0){$\blacksquare$}}
\put(1439,713){\makebox(0,0){$\blacksquare$}}
\put(1349,777){\makebox(0,0){$\blacksquare$}}
\put(785.0,609.0){\rule[-0.600pt]{78.774pt}{1.200pt}}
\sbox{\plotpoint}{\rule[-0.200pt]{0.400pt}{0.400pt}}%
\put(131.0,131.0){\rule[-0.200pt]{0.400pt}{175.375pt}}
\put(131.0,131.0){\rule[-0.200pt]{315.097pt}{0.400pt}}
\put(1439.0,131.0){\rule[-0.200pt]{0.400pt}{175.375pt}}
\put(131.0,859.0){\rule[-0.200pt]{315.097pt}{0.400pt}}
\end{picture}
}
\end{figure}
\end{itemize}
\end{frame}