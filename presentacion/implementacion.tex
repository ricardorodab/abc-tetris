\begin{frame}
\frametitle{Diseño de la implementación}
\begin{itemize}
\pause
\item Análisis del sistema.
\pause
\begin{itemize}
\item Mantener modelo apegado a la formalización del juego.
\item Desarrollar la heurística descrita por Dervis Karaboga.
\item Comunicación de la heurística con el juego de Tetris.
\pause
\begin{figure}
\hbox{\hspace{4.5em} \scalebox{.55}{\begin{figure}[H]
\centering
\begin{tikzpicture}
\umlsimpleclass[x=-6, y=1]{\_\_main\_\_.py}
\begin{umlpackage}[x=-6,y=0.7,scale=0.1, name=union]{}
\end{umlpackage}


\begin{umlpackage}[x=0 ,y=0]{test}
\dots
\end{umlpackage}

\begin{umlpackage}[x=0 ,y=0]{abejas-tetris}

\begin{umlpackage}[x=-3 ,y=-3]{abc}
\dots
\end{umlpackage}

\begin{umlpackage}[x=3 ,y=3]{tetris}
\dots
\end{umlpackage}

\umlclass[x=0, y=0]{Abejas\_Tetris}{
$\qquad\quad\dots$
}{
$\qquad\quad\dots$ 
}

\end{umlpackage}
\umlimport[geometry=|-,anchor1=0, anchor2=west, name=import]{abc}{test}
\umlimport[geometry=|-, name=import]{tetris}{test}
\umlimport[geometry=|-,anchor1=north, anchor2=90, name=import]{test}{union}
\end{tikzpicture}
\caption{Estructura básica del sistema.} \label{fig:uml-estructura}
\end{figure}}}
\end{figure}
\end{itemize}
\end{itemize}
\pause
\begin{center}
Orientación a Objetos
\end{center}
\end{frame}


\begin{frame}
\frametitle{Diseño de clases}
\pause
\LARGE
Tetris:
\normalsize

\begin{columns}
\column{0.5\textwidth}
\begin{itemize}
\item Punto.
\item Tipo.
\item Casilla.
\end{itemize}

\column{0.5\textwidth}
\begin{itemize}
\item Pieza.
\item Tablero.
\item Tetris.
\end{itemize}
\end{columns}
\pause

\vspace{10mm}

\LARGE
ABC:
\normalsize
\pause
\begin{columns}
\column{0.5\textwidth}
\begin{itemize}
\item Tipo\_Abeja.
\item Colmena.
\end{itemize}

\column{0.5\textwidth}
\begin{itemize}
\item Abeja:
\pause
\begin{itemize}
\item Exploradoras.
\item Trabajadoras.
\item \textcolor<6->{red}{Observadoras.}
\end{itemize}
\end{itemize}
\end{columns}

\end{frame}



\begin{frame}
\frametitle{Abejas observadoras}
\begin{itemize}
\item ¿Qué hacen exactamente las abejas observadoras en ABC? 
\begin{itemize}
\pause
\item Seleccionan una fuente $i$: 
\begin{displaymath}
  P_{i} = \frac{F(\theta_{i})}{\sum_{k=1}^{n} F(\theta_{k})}.
\end{displaymath}
\pause
\item Buscan una fuente vecina a $i$:
\begin{displaymath}
  \theta_{i}(c+1) = \theta_{i}(c) \pm \phi_{i}(c).
\end{displaymath}
\end{itemize}
\pause
\item ¿Cómo se proyecta una fuente en Tetris?
\pause
R= Para un juego $\langle B_{0}, P_{1}, ..., P_{p} \rangle$, una secuencia de
trayectorias $\Sigma$ es una secuencia $B_{0},\sigma_{1},B_{1},...,\sigma_{p},B_{p}$
tal que para cada $i$, la trayectoria de la pieza $P_{i}$ aplicado al tablero
$B_{i-1}$, genera el tablero $B_{i}$.
\end{itemize}
\end{frame}



\begin{frame}
\frametitle{Abejas observadoras}
\begin{itemize}
\item ¿Cómo se \textit{mueven} las abejas \textit{cerca} de una fuente?
\end{itemize}
\begin{figure}
\scalebox{.55}{\begin{figure}[H]
\centering
\begin{tikzpicture}[
    node distance = 21mm and 7mm,
    box/.style = {draw, minimum size=8mm, inner sep=0pt, outer sep=0pt, anchor=center},
      pin edge = {Straight Barb-, shorten <=1mm,semithick}]
        \node (h2) [box] {Izq};
		\node (h1) [box, left=of h2] {Der};
		\node (h3) [box, right=of h2]    {Gir};
		\node (h4) [box, right=of h3]    {Cae};
		\node (q_dots) [draw=none, right=of h4] {$\cdots$};
		\node (hk) [box, right=of q_dots] {$Mov_{m}$};
		
		\node (h5) [box, below=of h4] {Izq};
		\node (h32) [draw=none, left=of h5] {};
		\node (q_dots_two) [draw=none, right=of h5] {$\cdots$};
		\node (hl) [box, below=of hk] {$Mov_{n}$};
		
		
		\draw[dashed,-Straight Barb, shorten <=1mm, shorten >=1mm]
    (h3.south) edge [bend right]   (h5.north);	
		
		\draw[-Straight Barb, transform canvas={yshift= 0mm}]
    		(h1) edge   (h2);
    	\draw[-Straight Barb, transform canvas={yshift= 0mm}]
    		(h2) edge   (h3);
    	\draw[-Straight Barb, transform canvas={yshift= 0mm}]
    		(h3) edge   (h4);
    	\draw[-Straight Barb, transform canvas={yshift= 0mm}]
    		(h4) edge   (q_dots);    		
    	\draw[-Straight Barb, transform canvas={yshift= 0mm}]
    		(h5) edge   (q_dots_two);
    	\draw[-Straight Barb, transform canvas={yshift= 0mm}]
    		(q_dots_two) edge   (hl);
    	\draw[-Straight Barb, transform canvas={yshift= 0mm}]
    		(q_dots) edge   (hk);
    	\draw [decorate,decoration={brace,amplitude=6pt,raise=5ex}]
  (h1) -- (hk) node[midway,yshift=4em]{$\theta_{i}$};
  
 		 \draw [decorate,decoration={brace,amplitude=7pt,mirror,raise=5ex}]
  (h32) -- (hl) node[midway,yshift=-4em]{$\phi_{i}$};
    \end{tikzpicture}
\caption{Historial de movimientos dentro de un juego de Tetris.} \label{fig:listmov}
\end{figure}}
\end{figure}
\end{frame}