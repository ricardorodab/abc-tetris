\chapter{Introducción}

Desde el inicio del uso de las computadoras y el uso de la \textit{Máquina de
  Turing} como modelo de cómputo, uno de los principales objetivos ha sido el
tratar de resolver problemas de manera automatizada. A partir de la década de
los años cuarentas del siglo pasado y en pleno apogeo de la segunda guerra
mundial, los programadores afrontaron la gran complicación de usar un enfoque
para no orientar a las máquinas a calcular todos los posibles resultados sino
disminuir el número de operaciones. Por aquella tormentosa época, matemáticos y
criptógrafos no tuvieron opción más que reducir el espacio de búsqueda de sus
problemas para optimizar el tiempo de ejecución de sus (muy) limitadas
máquinas. Al reducir el espacio de búsqueda contribuyeron a encontrar soluciones
de una manera más eficiente~\cite{GalavizCasas}.

El modelo computacional que actualmente se usa puede resolver problemas
cotidianos de manera bastante eficiente, como son mostrar la letra de una canción en
uno de los buscadores más usado y más famoso~\cite{Harikumar}. Esta afirmación
parte de la idea de que existen problemas que son fáciles de entender pero
difíciles o imposibles de hacer que una computadora dé una respuesta;
está demostrado que problemas como \textit{The Halting problem}
\footnote{La función \texttt{HALT} toma de entrada un par $\langle \alpha, x
\rangle$ y regresa $1$ sí y sólo si la máquina de Turing $M_{\alpha}$ se
detiene dada la entrada $x$, en un número finito de pasos.} no pueden
ser resueltos por una máquina de Turing.
Existen incógnitas que pueden ser fácilmente enunciables como \textit{¿es $N$ un número
primo?}\footnote{Problema del número compuesto.} o \textit{¿puede una persona
recorrer $K$ lugares de forma eficiente sin repetir ninguno y al finalizar
regresar al lugar de origen?}\footnote{Problema del agente viajero.} pero al ser programadas,
pueden tardar años en ser solucionados por una máquina si no se les da
herramientas correctas de resolución~\cite{arora2009computational}.

Para todos aquellos problemas en los cuales crear un algoritmo y esperar
una solución óptima no es posible, ya sea por su complejidad o porque
simplemente no se sabe si existe un algoritmo eficientes, se usan otros métodos
que aunque posiblemente no sean los óptimos, nos ayudan a dar soluciones que
surgen con el objetivo de ser lo más efectivas posibles~\cite{Pearl1984}.
Algunas de las compañías de software de uso cotidiano y masivo, empresas tan
conocidas como de transporte~\cite{uber-heuristic} o comercio electrónico\footnote{Conocido como \textit{e-commerce}, incluye empresas como Uber,
Amazon, Netflix, Google, entre otras.}, han usado estas
alternativas como recurso para mejorar su impacto en el mercado~\cite{Linden},~\cite{Smith2017}.

Dada una problemática bien definida y que se pueda probar que exista 
dentro de la clase de complejidad
\textsl{NP}-completo o \textsl{NP}-duro y una heurística de optimización
combinatoria como las definidas en ~\cite{Pearl1984}, ¿cómo comparar si la
solución propuesta es buena?  Una forma es desarrollar la aplicación de la
heurística al problema y analizar los resultados obtenidos. Para analizar las
soluciones se deberá definir un objetivo a conseguir con los 
datos de entrada y un comportamiento que encuentre una solución 
al problema definido. De esta manera, se podría observar que los resultados son
eficiente al menos en algún punto de la ejecución.

El objetivo principal de este trabajo es tomar el problema de calcular los
posibles desenlaces de un juego de Tetris y seleccionar una serie de
movimientos adecuados, usando una heurística de solución genérica que realice
optimización combinatoria como la mencionada en~\cite{karaboga2005idea}.

Se usará la heurística de nombre \textit{Colonia de Abejas Artificiales} como
método de cálculo para los posibles desenlaces de cada iteración del problema.
La heurística consiste en dividir la búsqueda de posibles soluciones óptimas
usando la reducción del espacio con la ayuda de distintas estrategias que se
basan en las tácticas de un ente natural: conductas como el de
reducir el lugar geográfico de una colonia de abejas para localizar fuentes
de alimentos y mejorar el comportamiento de un panal.
El propósito de usar esta heurística es adaptar una función de costo que
se desarrollará sobre el contexto del juego de Tetris.

Se ha elegido como problema el juego de Tetris por las interesantes
conclusiones que se han obtenido del trabajo
en~\cite{DBLP:journals/corr/cs-CC-0210020} y por
la familiaridad que presenta para una gran cantidad de personas, por lo que
entender sus reglas y lógica no conlleva un reto mayor al del propósito del
presente trabajo. Los objetivos a optimizar por la heurística son: maximizar
el número de filas eliminadass mientras la computadora juega; maximizar el
número de piezas colocadas al finalizar el juego; maximizar el número de veces
que se realiza un ``tetris'' (que es cuando se eliminan simultáneamente cuatro
líneas); y minimizar la altura de la fila más alta en cualquier momento
durante el juego.

En el capítulo dos se hablará de diferentes tipos de problemas computacionales,
terminando con la explicación de un problema \textsl{NP}-completo que será el
punto de partida para explicar el problema a tratar en esta tesis;
Tetris, que será abordado en el capítulo cuatro y la solución a usar
será descrita en el capítulo tres. Tanto en el capítulo cinco como en el seis se
discutirá el ambiente y la implementación del problema mientras que en el siete
los resultados del desempeño de la solución propuesta. Para finalizar este
trabajo, las conclusiones se presentan en el capítulo ocho.

%%% Local Variables:
%%% mode: latex
%%% TeX-master: "../main"
%%% End:
