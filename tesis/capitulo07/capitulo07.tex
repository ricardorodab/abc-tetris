\chapter{Experimentación y resultados}

Para entender el comportamiento de la implementación al modificar los parámetros de 
entrada, es de gran importancia analizar los resultados y conducir a la heurística 
hacia un estado de evaluación que se considere positivo. Como consecuencia a 
la experimentación del sistema, el ajuste de los pesos y las funciones evaluadoras se deben 
ir refinando de tal manera que 
al observar el movimiento de las piezas e ignorar el funcionamiento interno 
del proyecto, un espectador podría llegar a coincidir en las decisiones tomadas 
por la colmena. Durante este capítulo se explicarán las distintas estrategias 
de evaluación y los resultados obtenidos de cada una de ellas.

\section{Funciones de comportamiento}

Como bien se mencionó, cada objeto creado desde la clase 
\texttt{Abeja} posee cuatro atributos que tienen 
la particularidad de ser funciones que serán llamadas por la colmena a la hora de 
iterar sobre cada uno de los tipos de abeja. Todas las funciones, excepto la de las 
abejas observadoras, reciben un objeto 
de tipo \texttt{T}, que es el tipo de la fuente, la función de las abejas 
observadoras recibe un objeto de tipo genérico \texttt{T} y un tipo \texttt{float}. 
El tipo \texttt{T} en la implementación de este proyecto es la clase \texttt{Tetris}.

La función que se almacena dentro del atributo 
$\texttt{\_busca\_fuente}:T \rightarrow T$, 
es usada por las abejas exploradoras y al igual que todas las funciones que poseen 
el resto de las abejas de la colmena, tiene la característica de que el parámetro 
\texttt{fuente} que recibe puede ser ignorado en el caso de que la función se desee 
definir de forma independiente a algún estado anterior como es usual en las heurísticas. 
En el caso particular del juego de Tetris, la función sí utiliza el parámetro 
como un punto de partida de las abejas y regresa un juego de Tetris de tipo 
\texttt{Tetris}, donde una pieza es colocada en alguna posición de forma aleatoria.

En el atributo $\texttt{\_explotar}:T\rightarrow float$ se guarda la función 
que todas las abejas empleadas usarán con la fuente que tienen asociada. 
Cada empleada de manera aleatoria buscará colocar una pieza dentro de la solución 
previa. Aunque es muy parecida a la función para buscar una fuente,  
esta función es diferente a $\texttt{\_busca\_fuente()}$ porque 
el valor regresado no es la fuente, sino la evaluación del resultado de Tetris. 
El estado del juego sí es modificado y la partida de Tetris ``avanza'' en el tiempo.

El atributo $\texttt{\_observadoras}:T \times float \rightarrow T$ posee un comportamiento 
un poco más complejo: en primer lugar clona el objeto que recibe para evitar 
modificar el estado de un objeto original asociado a alguna abeja empleada. Una vez teniendo 
un objeto clon, la abeja observadora realiza 
movimientos inversos dentro del historial reciente del juego de Tetris. En otras 
palabras, elimina pasos que la abeja empleada o exploradora hicieron para llegar 
al estado actual. Después de realizar los movimientos inversos, se le agrega de manera 
aleatoria nuevos movimientos hasta llegar al punto donde la pieza vuelva a ser 
fijada en el tablero. Las acciones de retirar movimientos y agregar nuevos puede 
verse como \textit{alejarse} de la fuente original para conseguir una fuente 
en una \textit{distancia} $\delta$. De vuelta en la colmena, si la solución a 
distancia $\delta$ es mejor que la función original, la nueva fuente será la 
localizada por la abeja observadora. Son las abejas observadoras las encargadas 
de explorar los mínimos locales mientras que las exploradoras buscan los mínimos 
globales.

La última función es la de néctar o también llamada función de costo o 
\textit{fitness}. Almacenada en $\texttt{\_nectar}: T \rightarrow float$, esta 
función evaluará los juegos de Tetris y les asignará un número $n$ que entre mayor 
sea, mejor será la cantidad de néctar y por lo tanto mejor la solución. Definir esta 
función conlleva una complejidad un poco mayor debido al número de variables 
que pueden influenciar en una partida de Tetris y que deben 
ser consideradas, ya que la definición tendrá un impacto directo sobre 
la calidad de resultados obtenidos; una función pobremente definida tendrá 
de forma obligada resultados pobres\footnote{En el caso de Tetris, un resultado 
pobre es considerado aquel en el que un número corto de iteraciones, lleva al estado 
de \textit{game\_\!over}.}. 

Debido que ninguna de las funciones mencionadas anteriormente están estrictamente 
definidas para resolver Tetris, se deben realizar pruebas y analizar los 
resultados para ajustar los parámetros que utilizan. 


\section{Métricas de desempeño}

El modo en que un jugador humano mide su desempeño jugando Tetris es generalmente usando 
tres factores de medición: el tiempo de juego, el puntaje obtenido y el número de filas que 
son removidas durante una partida. Para la implementación propuesta, el puntaje 
obtenido será sustituido por el valor de la función de costo y el tiempo de 
juego que será medido por el número de piezas colocadas en una partida. 

El propósito principal a optimizar será el número total de filas que son removidas 
por la colmena durante una partida completa de Tetris. Al optimizar las filas 
removidas, también hará posible que se incremente la función de costo y el número 
de casillas no ocupadas por fichas, aumentando el número de piezas que se puedan 
colocar en el tablero.

Existe en muchas versiones de Tetris modernos un modo de juego llamado 
\textit{40 lines} que consiste en remover 40 filas de una partida de Tetris 
para considerar que el jugador haya ganado~\cite{line-tetris}. En este trabajo, nos 
limitaremos a conseguir la mitad para considerar un historial de movimientos 
buenos dado una lista de fichas.

El número de piezas jugadas también se usa como una métrica para afirmar 
que un tablero es mejor sobre otro. Si ambos tableros generan la remoción 
de quince filas pero un tablero juega 20 piezas más que el otro, significa 
que las piezas presentan un mayor nivel de horizontalidad y por lo tanto existe 
una probabilidad mayor de que el tablero con mayor número de piezas pueda 
desaparecer una fila o al menos obtenga más tiempo de juego.

Las fichas a jugar deben ser generadas a partir del un programa en \emph{script}
que se encuentra en el archivo \texttt{./scripts/lista-fichas.sh}. El programa
generará un archivo en formato \texttt{.csv} que tendrá mil fichas aleatorias en
su representación de forma de cadena. Se debe correr el \emph{script} cada vez que se
desee generar una nueva combinación de fichas.

\section{Funciones de costo}

Como se ha explicado a lo largo de este trabajo, la función de costo tiene un 
gran impacto sobre los resultados que las heurísticas obtienen. Desafortunadamente 
son muchas las variables que influyen en una partida de Tetris que hay que 
considerar, tanto valores positivos como negativos que generan una calificación 
en el tablero. 

Para ejemplificar cómo una función de costo puede impactar directamente al juego, 
se puede suponer que existe $T$ un tablero vacío, $f_{1}$ y $f_{2}$ funciones definidas 
de la siguiente manera:

\begin{displaymath}
  \begin{array}{rcl}
    f_{1}: T & \rightarrow & float \\
    f_{2}: T & \rightarrow & float.
  \end{array}
\end{displaymath}


Para el siguiente ejemplo, se puede también suponer que las fichas a jugar por la colmena 
poseen el siguiente orden en específico: \texttt{L = \{I, Rg, Lg, Sq, 
I, T, Ls, Rs, I\}}.

Sea $f_{1}(T)$ tal que regresa un número mayor si cada pieza $P_{i}$ se 
encuentra inmediatamente a la derecha de cada pieza $P_{i-1}$. Si no existe 
lugar a la derecha debido a que la pieza $P_{i-1}$ está en el límite derecho del tablero, 
entonces las abejas considerarán una mejor solución el colocar la pieza  
en el extremo superior izquierdo. En la \cref{fig:mal-juego} se puede observar que 
la función no indica a las abejas que dejar casillas vacías entre niveles inferiores 
y superiores afecta negativamente a la partida. Si existen demasiadas casillas vacías, 
con un número bajo de piezas la probabilidad de perder crece de manera rápida. 

\begin{figure}[H]
\centering
% 4 es Sq
% 1 es I
% 2 es L(Lg) inversa y 6 es L(Rg)
% 3 es S(Rs) y Z es 7(Ls)
% 5 es T 
% 1, 6, 2, 4, 1, 5, 7, 3, 1
% I, Rg, Lg, Sq, I, T, Ls, Rs, I
\begin{tikzpicture}[omino/nodes/.style={shape=rectangle, rounded corners, inner sep=+0pt, minimum size=1cm-2\pgflinewidth}]
	\path [omino={at={-2,0}, rotate=0}][tetris=1];
	\path [omino={at={4,0}, rotate=0}][tetris=1];
    \path [omino={at={2,0}, rotate=0}][tetris=4];
	\path [omino={at={1,0}, rotate=0}][tetris=2];	
	\path [omino={at={0,0}, rotate=0}][tetris=6];
	\path [omino={at={5,0}, rotate=0}][tetris=5];
	\path [omino={at={7,1}, rotate=0}][tetris=7];
	\path [omino={at={-2,4}, rotate=0}][tetris=3];
	\path [omino={at={0,3}, rotate=0}][tetris=1];
%    \path [omino/at=0:1] [tetris=2];
%    \path [omino/at=0:2] [tetris=3];
%    \path [omino={at=0:3, rotate=-90, x mirror}][tetris=5];
%    \path [omino={at={5,1}, rotate=-90}][tetris=3];
%    \path [omino={at={5,2}, rotate=-90}][tetris=2];
%    \path [omino={at={4,2}, x mirror}][tetris=5];
%    \path [omino={at={1,3}}][tetris=4];
\end{tikzpicture}
\caption{Una mala función de costo deja muchas casillas atrapadas y cubiertas.} \label{fig:mal-juego}
\end{figure}

La función $f_{2}(T)$ se define de tal forma que si el tablero $T$ se encuentra 
vacío, se coloca la primera pieza en la posición más cercana al borde izquierdo. 
Si el tablero no está vacío entonces existe una casilla superior $h$ y el resultado 
de la función es mayor si la siguiente ficha colocada no supera dicho límite y 
se intentan llenar todas las casillas por debajo de $h$, dándole un peso mayor 
a las que se encuentren más cercanas al borde inferior y más cercanas al borde izquierdo.

\begin{figure}[H]
\centering
\begin{tikzpicture}[omino/nodes/.style={shape=rectangle, rounded corners, inner sep=+0pt, minimum size=1cm-2\pgflinewidth}]
	\path [omino={at={-1,-2}, rotate=0}][tetris=1];
	\path [omino={at={3,1}, rotate=90}][tetris=1];
    \path [omino={at={1,-1}, rotate=0}] [tetris=4];
	\path [omino={at={3,0}, rotate=180}][tetris=2];	
	\path [omino={at={0,0}, rotate=180}][tetris=6];
	\path [omino={at={6,-2}, rotate=90}][tetris=5];
	\path [omino={at={4,-1}, rotate=0}][tetris=3];
	\path [omino={at={8,-2}, rotate=0}][tetris=1];
	\path [omino={at={7,-2}, rotate=0}] [tetris=7];	
%    \path [omino/at=0:1] [tetris=2];
%    \path [omino/at=0:2] [tetris=3];
%    \path [omino={at=0:3, rotate=-90, x mirror}][tetris=5];
%    \path [omino={at={5,1}, rotate=-90}][tetris=3];
%    \path [omino={at={5,2}, rotate=-90}][tetris=2];
%    \path [omino={at={4,2}, x mirror}][tetris=5];
%    \path [omino={at={1,3}}][tetris=4];
\end{tikzpicture}
\caption{Una función que pareciera funcionar mejor al trata de cubrir todos los espacios.} \label{fig:medio-juego}
\end{figure}

Como se puede observar en la \cref{fig:medio-juego}, para la instancia de fichas 
\texttt{L} existen dos filas completamente llenas al finalizar la última 
iteración por lo que las casillas en dichas filas serían removidas y el 
propósito principal de la heurística que es jugar por más tiempo y eliminar el 
mayor número de filas se incrementaría. 

Aunque pareciera que la segunda función es mejor que la primera, el valor final 
de la heurística cambia si las fichas que contiene \texttt{L} fueran  
\texttt{L = \{I, I, Sq, Sq, Sq, Sq\}}. El resultado de la función $f_{1}(T)$ y 
$f_{2}(T)$ se pueden ver respectivamente en la \cref{fig:mal-juego-dos} 
y \cref{fig:medio-juego-dos}. Se observa que incluso con una función 
generadora de néctar que se considere no genere tan buenos resultados, como lo 
es $f_{1}$, con la entrada correcta puede funcionar mejor que otra función que genere 
mejores resultados con un conjunto mayor de valores de entrada.

\begin{figure}[H]
\centering
% 4 es Sq
% 1 es I
% 2 es L(Lg) inversa y 6 es L(Rg)
% 3 es S(Rs) y Z es 7(Ls)
% 5 es T 
% 1, 1, 4, 4, 4, 4
% I, I, Sq, Sq, Sq, Sq
\begin{tikzpicture}[omino/nodes/.style={shape=rectangle, rounded corners, inner sep=+0pt, minimum size=1cm-2\pgflinewidth}]
	\path [omino={at={-2,0}, rotate=0}][tetris=1];
	\path [omino={at={-1,0}, rotate=0}][tetris=1];
    \path [omino={at={0,0}, rotate=0}][tetris=4];
	\path [omino={at={2,0}, rotate=0}][tetris=4];	
	\path [omino={at={4,0}, rotate=0}][tetris=4];
	\path [omino={at={6,0}, rotate=0}][tetris=4];
\end{tikzpicture}
\caption{Una mala función puede dar buenos resultados si la entrada que recibe coincide con la lógica propuesta.} \label{fig:mal-juego-dos}
\end{figure}

\begin{figure}[H]
\centering
% 4 es Sq
% 1 es I
% 2 es L(Lg) inversa y 6 es L(Rg)
% 3 es S(Rs) y Z es 7(Ls)
% 5 es T 
% 1, 1, 4, 4, 4, 4
% I, I, Sq, Sq, Sq, Sq
\begin{tikzpicture}[omino/nodes/.style={shape=rectangle, rounded corners, inner sep=+0pt, minimum size=1cm-2\pgflinewidth}]
	\path [omino={at={-2,0}, rotate=0}][tetris=1];
	\path [omino={at={-1,0}, rotate=0}][tetris=1];
    \path [omino={at={0,0}, rotate=0}][tetris=4];
	\path [omino={at={0,2}, rotate=0}][tetris=4];	
	\path [omino={at={2,0}, rotate=0}][tetris=4];
	\path [omino={at={2,2}, rotate=0}][tetris=4];
\end{tikzpicture}
\caption{La función de costo que originalmente daba buenos resultados 
con una entrada específica no realiza mejoras considerables en la partida de un tablero de $10 \times 20$.} \label{fig:medio-juego-dos}
\end{figure}

\subsection{Filas entre pesos negativos}

La primera función de costo que pareció funcionar de una manera general
fue una división de valores de peso positivos entre negativos. Se hicieron una 
decena de pruebas utilizando exclusivamente un solo valor a optimizar 
sin obtener más de tres o cuatro filas removidas por juego pero la primer 
fórmula que se acerca al objetivo es:

\begin{displaymath}
    \frac{1 + \texttt{filas-removidas}}{\texttt{horizontalidad} + \texttt{atrapados} + \texttt{cubiertos} + \texttt{altura}} .
\end{displaymath}

Sea $y_{1}$ la altura de la casilla ocupada más arriba en el tablero y $y_{2}$ la casilla 
más abajo, la \texttt{horizontalidad} del tablero se define como $|y_{1} - y_{2}|$. 
Una casilla $(x,y)$ se considera \texttt{atrapada} si la casilla en la posición está vacía 
y sus vecinos inmediatos, las casillas en las posiciones $(x + 1, y)$, 
$(x, y + 1)$, $(x - 1, y)$ y $(x, y - 1)$ no se encuentran vacías. A diferencia de 
las casillas atrapadas, la casilla $(x_{n}, y_{i})$ se considera \texttt{cubierta} 
si existe una casilla ocupada en una posición $(x_{m},y_{j})$ tal que $j > i$ y $n = m$.

Esta función de néctar tiene la peculiaridad de tener cuatro variables de peso negativo 
y sólo un peso positivo, así que lo que las abejas hacen no es buscar un buen 
tablero, sino el tablero con el menor peso negativo. Si de paso las abejas 
encuentran una forma de eliminar filas, entonces se quedan con ese tablero. 
La remoción de filas sube considerablemente el valor de la fuente debido a 
que la variable \texttt{filas-removidas} es en realidad el número de filas que se 
han quitado durante la ejecución del juego por una constante muy alta.


\begin{figure}[H]
\centering
% 4 es Sq
% 1 es I
% 2 es L(Lg) inversa y 6 es L(Rg)
% 3 es S(Rs) y Z es 7(Ls)
% 5 es T 
% 1, 1, 4, 4, 4, 4
% I, I, Sq, Sq, Sq, Sq
\begin{tikzpicture}[omino/nodes/.style={shape=rectangle, rounded corners, inner sep=+0pt, minimum size=1cm-2\pgflinewidth}]
	\path [omino={at={0,0}, rotate=0}][tetris=2];
	\path [omino={at={4,0}, rotate=0}][tetris=4];
	\path [omino={at={5,2}, rotate=90}][tetris=1];
	\path [omino={at={9,0}, rotate=90}][tetris=1];	
	\path [omino={at={6,1}, rotate=0}][tetris=4];
	\path [omino={at={9,3}, rotate=180}][tetris=5];
\end{tikzpicture}
\caption{De izquierda a derecha las fichas son \texttt{Lg, I1, Sq1, Sq2, I2, T}.} \label{fig:eval-tablero}
\end{figure}

Si se analiza la \cref{fig:eval-tablero} se puede observar que la suma de los pesos negativos 
es bastante alta y su néctar no es muy bueno: 

\begin{enumerate}
\item No hay ninguna fila llena de casillas ocupadas por lo que la variable 
\texttt{filas-removidas} es igual a cero.

\item Existe una casilla atrapada entre las fichas \texttt{T, I} y \texttt{Sq2}.

\item Existen siete casillas consideradas como cubiertas, seis debajo de las fichas 
\texttt{Lg, I} más la casilla atrapada.

\item La horizontalidad tiene el valor de uno ya que la casilla ocupada más abajo 
alcanzable desde el tope del tablero tiene altura tres y la más arriba tiene 
altura cuatro. 

\item La altura total de la partida tiene el valor de cuatro.

\end{enumerate}

Suponiendo que todos los pesos negativos son multiplicados por peso uno, nos queda 
el siguiente valor del néctar de la fuente:

\begin{displaymath}
    \frac{1 + \texttt{0}}{\texttt{1} + \texttt{1} + \texttt{7} + \texttt{4}} = \frac{1}{13} \approx 0.077
\end{displaymath}

\begin{figure}[H]
\centering
% 4 es Sq
% 1 es I
% 2 es L(Lg) inversa y 6 es L(Rg)
% 3 es S(Rs) y Z es 7(Ls)
% 5 es T 
% 1, 1, 4, 4, 4, 4
% I, I, Sq, Sq, Sq, Sq
\begin{tikzpicture}[omino/nodes/.style={shape=rectangle, rounded corners, inner sep=+0pt, minimum size=1cm-2\pgflinewidth}]
	\path [omino={at={9,2}, rotate=180}][tetris=2];
	\path [omino={at={2,0}, rotate=0}][tetris=4];
	\path [omino={at={7,0}, rotate=90}][tetris=1];
	\path [omino={at={8,1}, rotate=90}][tetris=1];	
	\path [omino={at={0	,0}, rotate=0}][tetris=4];
	\path [omino={at={3,2}, rotate=-90}][tetris=5];
\end{tikzpicture}
\caption{Tablero que toma las piezas de la \cref{fig:eval-tablero} y la función de filas entre pesos negativos.} \label{fig:eval-tablero-dos}
\end{figure}

Un valor más aceptable por esta función de néctar sería el tablero de la 
\cref{fig:eval-tablero-dos} donde podemos observar que existen dos filas 
completamente llenas, lo que eleva el valor de la variable \texttt{filas-removidas} a 
$2 \times c$ donde $c$ es un peso constante. No se observan espacios entre las 
casillas ocupadas por lo que $\texttt{atrapados} = \texttt{cubiertos} = 0$. La 
altura así como el valor que mide la horizontalidad del tablero, es la misma antes 
y después de limpiar el tablero, es decir, remover las filas no cambia los valores de  
$\texttt{altura} = \texttt{horizontalidad} = 1$. Suponiendo una $c \geq 2$ y 
sustituyendo $c = 2$, la fórmula final de la función queda de la siguiente manera: 

\begin{displaymath}
    \frac{1 + (2 \times c)}{1 + 0 + 0 + 1} = \frac{1 + 4}{2} = \frac{5}{2} = 2.5
\end{displaymath}

Si en la colmena existieran ambos tableros tanto de la \cref{fig:eval-tablero} 
como de la \cref{fig:eval-tablero-dos}, las abejas observadoras escogerían ir a analizar 
fuentes vecinas de la fuente que tiene el néctar con valor $2.5$ o mayor y se 
mantendrían dichas fuentes con mayor probabilidad que el resto.

La función es un indicador de lo bueno o malo que puede llegar a ser un tablero 
sobre otro, sin embargo las variables que se utilizan pueden influenciar mucho 
o muy poco sobre un tablero ya que existen casos en los que una vez que una variable 
eleve su valor sobre la evaluación, minimice el impacto de las otras y las abejas 
simplemente las ignoren porque su optimización no tiene gran ventaja.

Las primeras pruebas muestran que la función trabaja de la forma que se creía que 
lo haría. En la \cref{fig:filaspesosresulprim} se ve un crecimiento de filas 
eliminadas del juego conforme las pruebas de las distintas semillas avanzan.

\begin{figure}[H]
% GNUPLOT: LaTeX picture
\setlength{\unitlength}{0.240900pt}
\ifx\plotpoint\undefined\newsavebox{\plotpoint}\fi
\sbox{\plotpoint}{\rule[-0.200pt]{0.400pt}{0.400pt}}%
\begin{picture}(1500,900)(0,0)
\sbox{\plotpoint}{\rule[-0.200pt]{0.400pt}{0.400pt}}%
\put(131.0,131.0){\rule[-0.200pt]{4.818pt}{0.400pt}}
\put(111,131){\makebox(0,0)[r]{$0$}}
\put(1419.0,131.0){\rule[-0.200pt]{4.818pt}{0.400pt}}
\put(131.0,313.0){\rule[-0.200pt]{4.818pt}{0.400pt}}
\put(111,313){\makebox(0,0)[r]{$5$}}
\put(1419.0,313.0){\rule[-0.200pt]{4.818pt}{0.400pt}}
\put(131.0,495.0){\rule[-0.200pt]{4.818pt}{0.400pt}}
\put(111,495){\makebox(0,0)[r]{$10$}}
\put(1419.0,495.0){\rule[-0.200pt]{4.818pt}{0.400pt}}
\put(131.0,677.0){\rule[-0.200pt]{4.818pt}{0.400pt}}
\put(111,677){\makebox(0,0)[r]{$15$}}
\put(1419.0,677.0){\rule[-0.200pt]{4.818pt}{0.400pt}}
\put(131.0,859.0){\rule[-0.200pt]{4.818pt}{0.400pt}}
\put(111,859){\makebox(0,0)[r]{$20$}}
\put(1419.0,859.0){\rule[-0.200pt]{4.818pt}{0.400pt}}
\put(131.0,131.0){\rule[-0.200pt]{0.400pt}{4.818pt}}
\put(131,90){\makebox(0,0){$0$}}
\put(131.0,839.0){\rule[-0.200pt]{0.400pt}{4.818pt}}
\put(393.0,131.0){\rule[-0.200pt]{0.400pt}{4.818pt}}
\put(393,90){\makebox(0,0){$200$}}
\put(393.0,839.0){\rule[-0.200pt]{0.400pt}{4.818pt}}
\put(654.0,131.0){\rule[-0.200pt]{0.400pt}{4.818pt}}
\put(654,90){\makebox(0,0){$400$}}
\put(654.0,839.0){\rule[-0.200pt]{0.400pt}{4.818pt}}
\put(916.0,131.0){\rule[-0.200pt]{0.400pt}{4.818pt}}
\put(916,90){\makebox(0,0){$600$}}
\put(916.0,839.0){\rule[-0.200pt]{0.400pt}{4.818pt}}
\put(1177.0,131.0){\rule[-0.200pt]{0.400pt}{4.818pt}}
\put(1177,90){\makebox(0,0){$800$}}
\put(1177.0,839.0){\rule[-0.200pt]{0.400pt}{4.818pt}}
\put(1439.0,131.0){\rule[-0.200pt]{0.400pt}{4.818pt}}
\put(1439,90){\makebox(0,0){$1000$}}
\put(1439.0,839.0){\rule[-0.200pt]{0.400pt}{4.818pt}}
\put(131.0,131.0){\rule[-0.200pt]{0.400pt}{175.375pt}}
\put(131.0,131.0){\rule[-0.200pt]{315.097pt}{0.400pt}}
\put(1439.0,131.0){\rule[-0.200pt]{0.400pt}{175.375pt}}
\put(131.0,859.0){\rule[-0.200pt]{315.097pt}{0.400pt}}
\put(30,495){\rotatebox{-270}{\makebox(0,0){Filas eliminadas}}
}\put(785,29){\makebox(0,0){Semillas probadas}}
\put(1279,818){\makebox(0,0)[r]{Filas entre pesos neg}}
\put(1299.0,818.0){\rule[-0.200pt]{24.090pt}{0.400pt}}
\put(132,313){\usebox{\plotpoint}}
\multiput(132.58,313.00)(0.492,1.530){21}{\rule{0.119pt}{1.300pt}}
\multiput(131.17,313.00)(12.000,33.302){2}{\rule{0.400pt}{0.650pt}}
\multiput(144.00,349.58)(0.809,0.499){143}{\rule{0.747pt}{0.120pt}}
\multiput(144.00,348.17)(116.450,73.000){2}{\rule{0.373pt}{0.400pt}}
\multiput(262.00,422.58)(0.899,0.499){143}{\rule{0.818pt}{0.120pt}}
\multiput(262.00,421.17)(129.303,73.000){2}{\rule{0.409pt}{0.400pt}}
\multiput(393.00,495.58)(1.078,0.500){361}{\rule{0.962pt}{0.120pt}}
\multiput(393.00,494.17)(390.004,182.000){2}{\rule{0.481pt}{0.400pt}}
\multiput(785.00,677.58)(3.008,0.499){215}{\rule{2.500pt}{0.120pt}}
\multiput(785.00,676.17)(648.811,109.000){2}{\rule{1.250pt}{0.400pt}}
\put(131.0,131.0){\rule[-0.200pt]{0.400pt}{175.375pt}}
\put(131.0,131.0){\rule[-0.200pt]{315.097pt}{0.400pt}}
\put(1439.0,131.0){\rule[-0.200pt]{0.400pt}{175.375pt}}
\put(131.0,859.0){\rule[-0.200pt]{315.097pt}{0.400pt}}
\end{picture}

\caption[short caption]{Relación de crecimiento de filas removidas por semillas usadas.}
\label{fig:filaspesosresulprim}
\end{figure} 


\subsection{\textit{Raining skyline} ponderado}

Una estrategia diferente es obtener el número de casillas en las que una 
pieza puede ser colocada y tratar de optimizar ese número. Al optimizar 
el número de casillas disponibles se garantiza que el juego le dará un 
peso mayor de manera automática 
a aquellos tableros en los que existan menos piezas; en otras palabras, 
la heurística con esta función de costo entregará una mejor evaluación 
a aquellos tablero en los que se remueven filas. A continuación se enuncia 
un ejemplo de cómo obtiene el valor del tablero esta función a la que 
se le da el nombre de \texttt{cuenta\_descubiertos()}:


\begin{figure}[H]
\centering
% 4 es Sq
% 1 es I
% 2 es L(Lg) inversa y 6 es L(Rg)
% 3 es S(Rs) y Z es 7(Ls)
% 5 es T 
% 1, 1, 4, 4, 4, 4
% I, I, Sq, Sq, Sq, Sq
\begin{tikzpicture}[omino/nodes/.style={shape=rectangle, rounded corners, inner sep=+0pt, minimum size=1cm-2\pgflinewidth}]
	
	\path [omino={at={-2,0}, rotate=0}][tetris=0];
	\path [omino={at={-1,0}, rotate=0}][tetris=0];
	\path [omino={at={0,0}, rotate=0}][tetris=0];
	\path [omino={at={1,0}, rotate=0}][tetris=0];
	\path [omino={at={2,0}, rotate=0}][tetris=0];
	\path [omino={at={3,0}, rotate=0}][tetris=0];
	\path [omino={at={4,0}, rotate=0}][tetris=0];
	\path [omino={at={5,0}, rotate=0}][tetris=0];
	\path [omino={at={6,0}, rotate=0}][tetris=0];
	\path [omino={at={7,0}, rotate=0}][tetris=0];
	
	\path [omino={at={-2,1}, rotate=0}][tetris=0];
	\path [omino={at={-1,1}, rotate=0}][tetris=0];
	\path [omino={at={0,1}, rotate=0}][tetris=0];	
	\path [omino={at={1,1}, rotate=0}][tetris=0];
	\path [omino={at={2,1}, rotate=0}][tetris=0];
	\path [omino={at={3,1}, rotate=0}][tetris=0];
	\path [omino={at={4,1}, rotate=0}][tetris=0];
	\path [omino={at={5,1}, rotate=0}][tetris=0];
	\path [omino={at={6,1}, rotate=0}][tetris=0];
	\path [omino={at={7,1}, rotate=0}][tetris=0];

	
	\path [omino={at={-2,2}, rotate=0}][tetris=0];
	\path [omino={at={-1,2}, rotate=0}][tetris=0];
	\path [omino={at={0,2}, rotate=0}][tetris=0];
	\path [omino={at={1,2}, rotate=0}][tetris=0];
	\path [omino={at={2,2}, rotate=0}][tetris=0];
	\path [omino={at={3,2}, rotate=0}][tetris=0];
	\path [omino={at={4,2}, rotate=0}][tetris=0];
	\path [omino={at={5,2}, rotate=0}][tetris=0];
	\path [omino={at={6,2}, rotate=0}][tetris=0];
	\path [omino={at={7,2}, rotate=0}][tetris=0];
	
	\path [omino={at={-2,3}, rotate=0}][tetris=0];
	\path [omino={at={-1,3}, rotate=0}][tetris=0];
	\path [omino={at={0,3}, rotate=0}][tetris=0];
	\path [omino={at={1,3}, rotate=0}][tetris=0];
	\path [omino={at={2,3}, rotate=0}][tetris=0];
	\path [omino={at={3,3}, rotate=0}][tetris=0];
	\path [omino={at={4,3}, rotate=0}][tetris=0];
	\path [omino={at={5,3}, rotate=0}][tetris=0];
	\path [omino={at={6,3}, rotate=0}][tetris=0];
	\path [omino={at={7,3}, rotate=0}][tetris=0];	
	
	\path [omino={at={-2,4}, rotate=0}][tetris=0];
	\path [omino={at={-1,4}, rotate=0}][tetris=0];
	\path [omino={at={0,4}, rotate=0}][tetris=0];
	\path [omino={at={1,4}, rotate=0}][tetris=0];
	\path [omino={at={2,4}, rotate=0}][tetris=0];
	\path [omino={at={3,4}, rotate=0}][tetris=0];
	\path [omino={at={4,4}, rotate=0}][tetris=0];
	\path [omino={at={5,4}, rotate=0}][tetris=0];
	\path [omino={at={6,4}, rotate=0}][tetris=0];
	\path [omino={at={7,4}, rotate=0}][tetris=0];	
	
	\path [omino={at={-2,5}, rotate=0}][tetris=0];
	\path [omino={at={-1,5}, rotate=0}][tetris=0];
	\path [omino={at={0,5}, rotate=0}][tetris=0];
	\path [omino={at={1,5}, rotate=0}][tetris=0];
	\path [omino={at={2,5}, rotate=0}][tetris=0];
	\path [omino={at={3,5}, rotate=0}][tetris=0];
	\path [omino={at={4,5}, rotate=0}][tetris=0];
	\path [omino={at={5,5}, rotate=0}][tetris=0];
	\path [omino={at={6,5}, rotate=0}][tetris=0];
	\path [omino={at={7,5}, rotate=0}][tetris=0];	
	
	\path [omino={at={-2,6}, rotate=0}][tetris=0];
	\path [omino={at={-1,6}, rotate=0}][tetris=0];
	\path [omino={at={0,6}, rotate=0}][tetris=0];
	\path [omino={at={1,6}, rotate=0}][tetris=0];
	\path [omino={at={2,6}, rotate=0}][tetris=0];
	\path [omino={at={3,6}, rotate=0}][tetris=0];
	\path [omino={at={4,6}, rotate=0}][tetris=0];
	\path [omino={at={5,6}, rotate=0}][tetris=0];
	\path [omino={at={6,6}, rotate=0}][tetris=0];
	\path [omino={at={7,6}, rotate=0}][tetris=0];			
	
	\path [omino={at={0,1}, rotate=90}][tetris=7];
	\path [omino={at={2,3}, rotate=90}][tetris=7];
	\path [omino={at={3,0}, rotate=0}][tetris=4];
	\path [omino={at={4,2}, rotate=0}][tetris=4];
	\path [omino={at={7,0}, rotate=90}][tetris=5];	
\end{tikzpicture}
\caption{Tablero aleatorio con cinco piezas de altura cuatro.} \label{fig:tablero-promedio-uno}
\end{figure}

Sea un tablero $T$ en un tiempo $t$, donde $T$ posee al menos una pieza, 
la evaluación de $T$ dependerá del número de casillas que posea el \textit{raining 
skyline}\footnote{``Horizonte de lluvia'' en inglés.} de las piezas colocadas 
en el tablero. La manera de obtener este \emph{raining 
skyline} es tomando una línea imaginaria que pasa por todas las casillas que no tienen 
a ninguna otra casilla ocupada arriba de ellas; en otras palabras, la línea se 
dibuja en la frontera superior del polinomio que crean todas las columnas con su 
casilla ocupada más alta. 
La posición de la casilla $cas \in C_{sup}$ es la altura más alta 
de cada columna del tablero, en caso de no existir una casilla ocupada en la columna, 
el valor de $y$ de $cas$ es $y=0$.

Al terminar de obtener todas las $cas \in C_{sup}$ de cada una de las columnas, la manera 
de obtener la cantidad $rs$ de casillas del \emph{raining skyline} se realiza de la siguiente forma:

\begin{displaymath}
  rs = \sum_{i=1,j=1}^{ancho, alto} j \enskip | \enskip (i,j) \in C_{sup}.
\end{displaymath} 
 
Una vez obtenido el número $rs$ de casillas dentro del \emph{raining skyline}, se sabe que si 
existen $n$ casillas totales en el tablero $T$, el número de casillas 
fuera del \emph{raining skyline} $cf$ está dado por $cf = n - rs$. Una vez obtenido el 
valor $cf$, se puede devolver el valor del néctar del tablero de la siguiente forma: 

\begin{displaymath}
  f(T) = \frac{cf}{n}.
\end{displaymath} 

Para obtener el mayor provecho de esta función, es necesario que la bandera de 
configuración \texttt{LIMPIEZA} que se encuentra en archivo de configuraciones 
globales esté desactivada (con valor \texttt{False}) debido a que si está activada 
cada tablero eliminará piezas cada vez que se calcule el néctar y de esta forma se 
producirá una pérdida de información al tener menos casillas en el \emph{raining skyline}.


\begin{figure}[H]
\centering
% 4 es Sq
% 1 es I
% 2 es L(Lg) inversa y 6 es L(Rg)
% 3 es S(Rs) y Z es 7(Ls)
% 5 es T 
% 1, 1, 4, 4, 4, 4
% I, I, Sq, Sq, Sq, Sq
\begin{tikzpicture}[omino/nodes/.style={shape=rectangle, rounded corners, inner sep=+0pt, minimum size=1cm-2\pgflinewidth}]
	%\path [omino={at={0,0}, rotate=0}][tetris=8];
	%\path [omino={at={1,1}, rotate=0}][tetris=9];
	
	\path [omino={at={-2,0}, rotate=0}][tetris=9];
	\path [omino={at={-1,0}, rotate=0}][tetris=9];
	\path [omino={at={0,0}, rotate=0}][tetris=9];
	\path [omino={at={1,0}, rotate=0}][tetris=9];
	\path [omino={at={2,0}, rotate=0}][tetris=9];
	\path [omino={at={3,0}, rotate=0}][tetris=9];
	\path [omino={at={4,0}, rotate=0}][tetris=9];
	\path [omino={at={5,0}, rotate=0}][tetris=9];
	\path [omino={at={6,0}, rotate=0}][tetris=9];
	\path [omino={at={7,0}, rotate=0}][tetris=9];
	
	\path [omino={at={-2,1}, rotate=0}][tetris=8];
	\path [omino={at={-1,1}, rotate=0}][tetris=9];
	\path [omino={at={0,1}, rotate=0}][tetris=9];	
	\path [omino={at={1,1}, rotate=0}][tetris=9];
	\path [omino={at={2,1}, rotate=0}][tetris=9];
	\path [omino={at={3,1}, rotate=0}][tetris=9];
	\path [omino={at={4,1}, rotate=0}][tetris=9];
	\path [omino={at={5,1}, rotate=0}][tetris=9];
	\path [omino={at={6,1}, rotate=0}][tetris=9];
	\path [omino={at={7,1}, rotate=0}][tetris=8];

	
	\path [omino={at={-2,2}, rotate=0}][tetris=8];
	\path [omino={at={-1,2}, rotate=0}][tetris=8];
	\path [omino={at={0,2}, rotate=0}][tetris=9];
	\path [omino={at={1,2}, rotate=0}][tetris=9];
	\path [omino={at={2,2}, rotate=0}][tetris=9];
	\path [omino={at={3,2}, rotate=0}][tetris=8];
	\path [omino={at={4,2}, rotate=0}][tetris=9];
	\path [omino={at={5,2}, rotate=0}][tetris=9];
	\path [omino={at={6,2}, rotate=0}][tetris=8];
	\path [omino={at={7,2}, rotate=0}][tetris=8];
	
	\path [omino={at={-2,3}, rotate=0}][tetris=8];
	\path [omino={at={-1,3}, rotate=0}][tetris=8];
	\path [omino={at={0,3}, rotate=0}][tetris=8];
	\path [omino={at={1,3}, rotate=0}][tetris=9];
	\path [omino={at={2,3}, rotate=0}][tetris=9];
	\path [omino={at={3,3}, rotate=0}][tetris=8];
	\path [omino={at={4,3}, rotate=0}][tetris=9];
	\path [omino={at={5,3}, rotate=0}][tetris=9];
	\path [omino={at={6,3}, rotate=0}][tetris=8];
	\path [omino={at={7,3}, rotate=0}][tetris=8];	
	
	\path [omino={at={-2,4}, rotate=0}][tetris=8];
	\path [omino={at={-1,4}, rotate=0}][tetris=8];
	\path [omino={at={0,4}, rotate=0}][tetris=8];
	\path [omino={at={1,4}, rotate=0}][tetris=8];
	\path [omino={at={2,4}, rotate=0}][tetris=8];
	\path [omino={at={3,4}, rotate=0}][tetris=8];
	\path [omino={at={4,4}, rotate=0}][tetris=8];
	\path [omino={at={5,4}, rotate=0}][tetris=8];
	\path [omino={at={6,4}, rotate=0}][tetris=8];
	\path [omino={at={7,4}, rotate=0}][tetris=8];	
	
	\path [omino={at={-2,5}, rotate=0}][tetris=8];
	\path [omino={at={-1,5}, rotate=0}][tetris=8];
	\path [omino={at={0,5}, rotate=0}][tetris=8];
	\path [omino={at={1,5}, rotate=0}][tetris=8];
	\path [omino={at={2,5}, rotate=0}][tetris=8];
	\path [omino={at={3,5}, rotate=0}][tetris=8];
	\path [omino={at={4,5}, rotate=0}][tetris=8];
	\path [omino={at={5,5}, rotate=0}][tetris=8];
	\path [omino={at={6,5}, rotate=0}][tetris=8];
	\path [omino={at={7,5}, rotate=0}][tetris=8];	
	
	\path [omino={at={-2,6}, rotate=0}][tetris=8];
	\path [omino={at={-1,6}, rotate=0}][tetris=8];
	\path [omino={at={0,6}, rotate=0}][tetris=8];
	\path [omino={at={1,6}, rotate=0}][tetris=8];
	\path [omino={at={2,6}, rotate=0}][tetris=8];
	\path [omino={at={3,6}, rotate=0}][tetris=8];
	\path [omino={at={4,6}, rotate=0}][tetris=8];
	\path [omino={at={5,6}, rotate=0}][tetris=8];
	\path [omino={at={6,6}, rotate=0}][tetris=8];
	\path [omino={at={7,6}, rotate=0}][tetris=8];		
	
		
\end{tikzpicture}
\caption{Ejemplo de cuál es el raining skyline (en rosado) del tablero en la \cref{fig:tablero-promedio-uno}.} \label{fig:raining skyline}
\end{figure}

En un comienzo se experimentó considerando que todas las casillas 
tienen en principio el mismo peso pero los resultados durante la experimentación 
no hicieron que superara a 5 el número de filas eliminadas durante la ejecución.
Se probaron $5,000$ semillas diferentes pero usando sólo la función como se  
definió, la experimentación mostró casos en los que incluso un tablero posee mejor 
néctar con un menor número de filas removidas.

\begin{figure}[h]
% GNUPLOT: LaTeX picture
\setlength{\unitlength}{0.240900pt}
\ifx\plotpoint\undefined\newsavebox{\plotpoint}\fi
\sbox{\plotpoint}{\rule[-0.200pt]{0.400pt}{0.400pt}}%
\begin{picture}(1500,900)(0,0)
\sbox{\plotpoint}{\rule[-0.200pt]{0.400pt}{0.400pt}}%
\put(111.0,252.0){\rule[-0.200pt]{4.818pt}{0.400pt}}
\put(91,252){\makebox(0,0)[r]{$0$}}
\put(1419.0,252.0){\rule[-0.200pt]{4.818pt}{0.400pt}}
\put(111.0,859.0){\rule[-0.200pt]{4.818pt}{0.400pt}}
\put(91,859){\makebox(0,0)[r]{$5$}}
\put(1419.0,859.0){\rule[-0.200pt]{4.818pt}{0.400pt}}
\put(111.0,131.0){\rule[-0.200pt]{0.400pt}{4.818pt}}
\put(111,90){\makebox(0,0){$0$}}
\put(111.0,839.0){\rule[-0.200pt]{0.400pt}{4.818pt}}
\put(377.0,131.0){\rule[-0.200pt]{0.400pt}{4.818pt}}
\put(377,90){\makebox(0,0){$1000$}}
\put(377.0,839.0){\rule[-0.200pt]{0.400pt}{4.818pt}}
\put(642.0,131.0){\rule[-0.200pt]{0.400pt}{4.818pt}}
\put(642,90){\makebox(0,0){$2000$}}
\put(642.0,839.0){\rule[-0.200pt]{0.400pt}{4.818pt}}
\put(908.0,131.0){\rule[-0.200pt]{0.400pt}{4.818pt}}
\put(908,90){\makebox(0,0){$3000$}}
\put(908.0,839.0){\rule[-0.200pt]{0.400pt}{4.818pt}}
\put(1173.0,131.0){\rule[-0.200pt]{0.400pt}{4.818pt}}
\put(1173,90){\makebox(0,0){$4000$}}
\put(1173.0,839.0){\rule[-0.200pt]{0.400pt}{4.818pt}}
\put(1439.0,131.0){\rule[-0.200pt]{0.400pt}{4.818pt}}
\put(1439,90){\makebox(0,0){$5000$}}
\put(1439.0,839.0){\rule[-0.200pt]{0.400pt}{4.818pt}}
\put(111.0,131.0){\rule[-0.200pt]{0.400pt}{175.375pt}}
\put(111.0,131.0){\rule[-0.200pt]{319.915pt}{0.400pt}}
\put(1439.0,131.0){\rule[-0.200pt]{0.400pt}{175.375pt}}
\put(111.0,859.0){\rule[-0.200pt]{319.915pt}{0.400pt}}
\put(30,495){\rotatebox{-270}{\makebox(0,0){Filas eliminadas}}
}\put(775,29){\makebox(0,0){Semillas probadas}}
\put(111,616){\usebox{\plotpoint}}
\put(110.67,495){\rule{0.400pt}{29.149pt}}
\multiput(110.17,555.50)(1.000,-60.500){2}{\rule{0.400pt}{14.574pt}}
\put(112.17,252){\rule{0.400pt}{48.700pt}}
\multiput(111.17,393.92)(2.000,-141.921){2}{\rule{0.400pt}{24.350pt}}
\multiput(128.00,252.58)(1.188,0.500){483}{\rule{1.050pt}{0.120pt}}
\multiput(128.00,251.17)(574.821,243.000){2}{\rule{0.525pt}{0.400pt}}
\multiput(705.00,495.58)(0.687,0.500){483}{\rule{0.650pt}{0.120pt}}
\multiput(705.00,494.17)(332.651,243.000){2}{\rule{0.325pt}{0.400pt}}
\put(114.0,252.0){\rule[-0.200pt]{3.373pt}{0.400pt}}
\put(111,616){\makebox(0,0){$+$}}
\put(112,495){\makebox(0,0){$+$}}
\put(114,252){\makebox(0,0){$+$}}
\put(123,252){\makebox(0,0){$+$}}
\put(128,252){\makebox(0,0){$+$}}
\put(705,495){\makebox(0,0){$+$}}
\put(1039,738){\makebox(0,0){$+$}}
\put(1439,738){\makebox(0,0){$+$}}
\put(1039.0,738.0){\rule[-0.200pt]{96.360pt}{0.400pt}}
\put(111.0,131.0){\rule[-0.200pt]{0.400pt}{175.375pt}}
\put(111.0,131.0){\rule[-0.200pt]{319.915pt}{0.400pt}}
\put(1439.0,131.0){\rule[-0.200pt]{0.400pt}{175.375pt}}
\put(111.0,859.0){\rule[-0.200pt]{319.915pt}{0.400pt}}
\end{picture}

\caption[short caption]{Relación de crecimiento de filas eliminadas por semillas usando 
la función de \emph{raining skyline} sin ponderar.}
\label{fig:rainingskyline}
\end{figure}

El siguiente paso fue asignarle a cada casilla, vacía y ocupada un valor que es 
directamente proporcional a su posición $y$ en el tablero. Este valor que le es 
asignado tiene el propósito de indicarle a las abejas dos cosas sobre el tablero:

\begin{enumerate}
\item Entre más abajo coloquen las abejas la pieza, menor peso restarán al conjunto de piezas 
disponibles. 

\item La altura del tablero es un indicador de lo ``saludable'' que es la decisión 
tomada. 
\end{enumerate}

La función de costo $f(x)$ entonces es redefinida de la siguiente manera para que 
pueda considerar no sólo la altura sino también el peso de cada una de las casillas que 
se encuentran dentro de la altura:

\begin{displaymath}
  rs = \sum_{i=1,j=1}^{ancho, alto} j \enskip | \enskip j > b \enskip \land \enskip (i,b) \in C_{sup} 
  \end{displaymath} 

  \begin{displaymath}
    f(T) = \frac{rs}{\sum\limits_{i=1}^{alto} (ancho \times i)}.
\end{displaymath} 


La función ahora además de darle prioridad a tableros en los que 
se trata de conseguir la menor área de casillas ocupadas, de estos tableros 
se considerará mejor solución aquellos en los que las casillas que se 
tengan que ocupar sean casillas inferiores. 

Los primeros resultados de esta 
función fueron mejores que las de su primer versión. Si bien los movimientos 
mejoraron el tablero en función del tiempo considerablemente, lo hizo de manera 
gradual con base a el número de semillas que se probaron como se puede ver en 
la \cref{fig:rainingskylinenopdos}

\begin{figure}[h]
% GNUPLOT: LaTeX picture
\setlength{\unitlength}{0.240900pt}
\ifx\plotpoint\undefined\newsavebox{\plotpoint}\fi
\sbox{\plotpoint}{\rule[-0.200pt]{0.400pt}{0.400pt}}%
\begin{picture}(1500,900)(0,0)
\sbox{\plotpoint}{\rule[-0.200pt]{0.400pt}{0.400pt}}%
\put(131.0,197.0){\rule[-0.200pt]{4.818pt}{0.400pt}}
\put(111,197){\makebox(0,0)[r]{$0$}}
\put(1419.0,197.0){\rule[-0.200pt]{4.818pt}{0.400pt}}
\put(131.0,363.0){\rule[-0.200pt]{4.818pt}{0.400pt}}
\put(111,363){\makebox(0,0)[r]{$5$}}
\put(1419.0,363.0){\rule[-0.200pt]{4.818pt}{0.400pt}}
\put(131.0,528.0){\rule[-0.200pt]{4.818pt}{0.400pt}}
\put(111,528){\makebox(0,0)[r]{$10$}}
\put(1419.0,528.0){\rule[-0.200pt]{4.818pt}{0.400pt}}
\put(131.0,694.0){\rule[-0.200pt]{4.818pt}{0.400pt}}
\put(111,694){\makebox(0,0)[r]{$15$}}
\put(1419.0,694.0){\rule[-0.200pt]{4.818pt}{0.400pt}}
\put(131.0,859.0){\rule[-0.200pt]{4.818pt}{0.400pt}}
\put(111,859){\makebox(0,0)[r]{$20$}}
\put(1419.0,859.0){\rule[-0.200pt]{4.818pt}{0.400pt}}
\put(131.0,131.0){\rule[-0.200pt]{0.400pt}{4.818pt}}
\put(131,90){\makebox(0,0){$0$}}
\put(131.0,839.0){\rule[-0.200pt]{0.400pt}{4.818pt}}
\put(393.0,131.0){\rule[-0.200pt]{0.400pt}{4.818pt}}
\put(393,90){\makebox(0,0){$1000$}}
\put(393.0,839.0){\rule[-0.200pt]{0.400pt}{4.818pt}}
\put(654.0,131.0){\rule[-0.200pt]{0.400pt}{4.818pt}}
\put(654,90){\makebox(0,0){$2000$}}
\put(654.0,839.0){\rule[-0.200pt]{0.400pt}{4.818pt}}
\put(916.0,131.0){\rule[-0.200pt]{0.400pt}{4.818pt}}
\put(916,90){\makebox(0,0){$3000$}}
\put(916.0,839.0){\rule[-0.200pt]{0.400pt}{4.818pt}}
\put(1177.0,131.0){\rule[-0.200pt]{0.400pt}{4.818pt}}
\put(1177,90){\makebox(0,0){$4000$}}
\put(1177.0,839.0){\rule[-0.200pt]{0.400pt}{4.818pt}}
\put(1439.0,131.0){\rule[-0.200pt]{0.400pt}{4.818pt}}
\put(1439,90){\makebox(0,0){$5000$}}
\put(1439.0,839.0){\rule[-0.200pt]{0.400pt}{4.818pt}}
\put(131.0,131.0){\rule[-0.200pt]{0.400pt}{175.375pt}}
\put(131.0,131.0){\rule[-0.200pt]{315.097pt}{0.400pt}}
\put(1439.0,131.0){\rule[-0.200pt]{0.400pt}{175.375pt}}
\put(131.0,859.0){\rule[-0.200pt]{315.097pt}{0.400pt}}
\put(30,495){\rotatebox{-270}{\makebox(0,0){Filas eliminadas}}
}\put(785,29){\makebox(0,0){Semillas probadas}}
\put(1279,818){\makebox(0,0)[r]{Función raining-skyline}}
\put(1299.0,818.0){\rule[-0.200pt]{24.090pt}{0.400pt}}
\put(131,296){\usebox{\plotpoint}}
\put(130.67,263){\rule{0.400pt}{7.950pt}}
\multiput(130.17,279.50)(1.000,-16.500){2}{\rule{0.400pt}{3.975pt}}
\put(132.17,197){\rule{0.400pt}{13.300pt}}
\multiput(131.17,235.40)(2.000,-38.395){2}{\rule{0.400pt}{6.650pt}}
\multiput(148.00,197.58)(4.323,0.499){129}{\rule{3.542pt}{0.120pt}}
\multiput(148.00,196.17)(560.648,66.000){2}{\rule{1.771pt}{0.400pt}}
\multiput(716.00,263.58)(2.465,0.499){131}{\rule{2.064pt}{0.120pt}}
\multiput(716.00,262.17)(324.716,67.000){2}{\rule{1.032pt}{0.400pt}}
\put(134.0,197.0){\rule[-0.200pt]{3.373pt}{0.400pt}}
\put(1045.0,330.0){\rule[-0.200pt]{94.915pt}{0.400pt}}
\sbox{\plotpoint}{\rule[-0.500pt]{1.000pt}{1.000pt}}%
\sbox{\plotpoint}{\rule[-0.200pt]{0.400pt}{0.400pt}}%
\put(1279,777){\makebox(0,0)[r]{Función raining-skyline ponderada}}
\sbox{\plotpoint}{\rule[-0.500pt]{1.000pt}{1.000pt}}%
\multiput(1299,777)(20.756,0.000){5}{\usebox{\plotpoint}}
\put(1399,777){\usebox{\plotpoint}}
\put(131,363){\usebox{\plotpoint}}
\multiput(131,363)(0.208,-20.754){5}{\usebox{\plotpoint}}
\put(135.77,263.00){\usebox{\plotpoint}}
\multiput(143,263)(1.568,-20.696){3}{\usebox{\plotpoint}}
\multiput(148,197)(19.588,6.863){29}{\usebox{\plotpoint}}
\multiput(716,396)(17.783,10.702){19}{\usebox{\plotpoint}}
\multiput(1045,594)(20.118,5.106){19}{\usebox{\plotpoint}}
\put(1439,694){\usebox{\plotpoint}}
\put(131,363){\makebox(0,0){$\circ$}}
\put(132,263){\makebox(0,0){$\circ$}}
\put(134,263){\makebox(0,0){$\circ$}}
\put(143,263){\makebox(0,0){$\circ$}}
\put(148,197){\makebox(0,0){$\circ$}}
\put(716,396){\makebox(0,0){$\circ$}}
\put(1045,594){\makebox(0,0){$\circ$}}
\put(1439,694){\makebox(0,0){$\circ$}}
\put(1349,777){\makebox(0,0){$\circ$}}
\sbox{\plotpoint}{\rule[-0.200pt]{0.400pt}{0.400pt}}%
\put(131.0,131.0){\rule[-0.200pt]{0.400pt}{175.375pt}}
\put(131.0,131.0){\rule[-0.200pt]{315.097pt}{0.400pt}}
\put(1439.0,131.0){\rule[-0.200pt]{0.400pt}{175.375pt}}
\put(131.0,859.0){\rule[-0.200pt]{315.097pt}{0.400pt}}
\end{picture}

\caption[short caption]{Relación de crecimiento de filas eliminadas por semillas usando 
la función de \emph{raining skyline} con y sin ponderar las casillas.}
\label{fig:rainingskylinenopdos}
\end{figure} 

Aunque el resultado se acerca al objetivo (20 filas eliminadas), las pruebas 
no se acercan al resultado esperado tan rápidamente. Algunas alternativas son
aumentar el tamaño de la colmena o la distancia que las abejas observadoras 
recorren pero ambas opciones tienen la desventaja que el tiempo por semilla se 
incrementa. 

Se puede observar en la \cref{fig:rainingskylinenopdos} que el primer resultado 
obtenido por la función elimina cinco filas pero pronto encuentra una semilla 
que le genera un mejor resultado y las filas que son eliminadas caen a sólo dos. 
Aunque la función trabaja correctamente y su valor de evaluación en efecto 
puede llegar a ser mayor independientemente del número de filas que retira, 
el quitar filas es el objetivo que se desea optimizar.

Para solucionar el problema de la disminución de filas eliminadas, 
lo que se hace es modificar de nuevo la función de tal manera que el obtener 
un número mayor de filas eliminadas, impacte positivamente al resultado de 
la evaluación. La obtención de los pesos y la fórmula para obtener el \emph{raining 
skyline} no se desea modificar por lo que sólo agregaremos una nueva variable; 
sea $fe$ el número de filas eliminadas en un tablero de Tetris, 
la nueva función queda de la siguiente manera:

 \begin{displaymath}
    f(T) = \left(\frac{rs}{\sum\limits_{i=1}^{alto} (ancho \times i)}\right) \times (1 + fe).
\end{displaymath} 

Donde $rs$ no cambia su definición de la segunda versión de la función. 
Inmediatamente al poner a prueba la nueva función, con pocas semillas se 
observa un impacto favorable muy notable en el desempeño de la función de 
costo. Se puede observar y comparar el resultado de esta función en la 
\cref{fig:rainingskylinenoptres}.

\begin{figure}[h]
% GNUPLOT: LaTeX picture
\setlength{\unitlength}{0.240900pt}
\ifx\plotpoint\undefined\newsavebox{\plotpoint}\fi
\sbox{\plotpoint}{\rule[-0.200pt]{0.400pt}{0.400pt}}%
\begin{picture}(1500,900)(0,0)
\sbox{\plotpoint}{\rule[-0.200pt]{0.400pt}{0.400pt}}%
\put(131.0,170.0){\rule[-0.200pt]{4.818pt}{0.400pt}}
\put(111,170){\makebox(0,0)[r]{$0$}}
\put(1419.0,170.0){\rule[-0.200pt]{4.818pt}{0.400pt}}
\put(131.0,269.0){\rule[-0.200pt]{4.818pt}{0.400pt}}
\put(111,269){\makebox(0,0)[r]{$5$}}
\put(1419.0,269.0){\rule[-0.200pt]{4.818pt}{0.400pt}}
\put(131.0,367.0){\rule[-0.200pt]{4.818pt}{0.400pt}}
\put(111,367){\makebox(0,0)[r]{$10$}}
\put(1419.0,367.0){\rule[-0.200pt]{4.818pt}{0.400pt}}
\put(131.0,465.0){\rule[-0.200pt]{4.818pt}{0.400pt}}
\put(111,465){\makebox(0,0)[r]{$15$}}
\put(1419.0,465.0){\rule[-0.200pt]{4.818pt}{0.400pt}}
\put(131.0,564.0){\rule[-0.200pt]{4.818pt}{0.400pt}}
\put(111,564){\makebox(0,0)[r]{$20$}}
\put(1419.0,564.0){\rule[-0.200pt]{4.818pt}{0.400pt}}
\put(131.0,662.0){\rule[-0.200pt]{4.818pt}{0.400pt}}
\put(111,662){\makebox(0,0)[r]{$25$}}
\put(1419.0,662.0){\rule[-0.200pt]{4.818pt}{0.400pt}}
\put(131.0,761.0){\rule[-0.200pt]{4.818pt}{0.400pt}}
\put(111,761){\makebox(0,0)[r]{$30$}}
\put(1419.0,761.0){\rule[-0.200pt]{4.818pt}{0.400pt}}
\put(131.0,859.0){\rule[-0.200pt]{4.818pt}{0.400pt}}
\put(111,859){\makebox(0,0)[r]{$35$}}
\put(1419.0,859.0){\rule[-0.200pt]{4.818pt}{0.400pt}}
\put(131.0,131.0){\rule[-0.200pt]{0.400pt}{4.818pt}}
\put(131,90){\makebox(0,0){$0$}}
\put(131.0,839.0){\rule[-0.200pt]{0.400pt}{4.818pt}}
\put(393.0,131.0){\rule[-0.200pt]{0.400pt}{4.818pt}}
\put(393,90){\makebox(0,0){$1000$}}
\put(393.0,839.0){\rule[-0.200pt]{0.400pt}{4.818pt}}
\put(654.0,131.0){\rule[-0.200pt]{0.400pt}{4.818pt}}
\put(654,90){\makebox(0,0){$2000$}}
\put(654.0,839.0){\rule[-0.200pt]{0.400pt}{4.818pt}}
\put(916.0,131.0){\rule[-0.200pt]{0.400pt}{4.818pt}}
\put(916,90){\makebox(0,0){$3000$}}
\put(916.0,839.0){\rule[-0.200pt]{0.400pt}{4.818pt}}
\put(1177.0,131.0){\rule[-0.200pt]{0.400pt}{4.818pt}}
\put(1177,90){\makebox(0,0){$4000$}}
\put(1177.0,839.0){\rule[-0.200pt]{0.400pt}{4.818pt}}
\put(1439.0,131.0){\rule[-0.200pt]{0.400pt}{4.818pt}}
\put(1439,90){\makebox(0,0){$5000$}}
\put(1439.0,839.0){\rule[-0.200pt]{0.400pt}{4.818pt}}
\put(131.0,131.0){\rule[-0.200pt]{0.400pt}{175.375pt}}
\put(131.0,131.0){\rule[-0.200pt]{315.097pt}{0.400pt}}
\put(1439.0,131.0){\rule[-0.200pt]{0.400pt}{175.375pt}}
\put(131.0,859.0){\rule[-0.200pt]{315.097pt}{0.400pt}}
\put(785,525){\makebox(0,0)[l]{Objetivo}}
\put(131,564){\line(1,0){1308}}
\put(30,495){\rotatebox{-270}{\makebox(0,0){Filas eliminadas}}
}\put(785,29){\makebox(0,0){Semillas probadas}}
\put(1279,818){\makebox(0,0)[r]{Función raining-skyline}}
\put(1299.0,818.0){\rule[-0.200pt]{24.090pt}{0.400pt}}
\put(131,229){\usebox{\plotpoint}}
\put(130.67,210){\rule{0.400pt}{4.577pt}}
\multiput(130.17,219.50)(1.000,-9.500){2}{\rule{0.400pt}{2.289pt}}
\put(132.17,170){\rule{0.400pt}{8.100pt}}
\multiput(131.17,193.19)(2.000,-23.188){2}{\rule{0.400pt}{4.050pt}}
\multiput(148.00,170.58)(7.158,0.498){77}{\rule{5.780pt}{0.120pt}}
\multiput(148.00,169.17)(556.003,40.000){2}{\rule{2.890pt}{0.400pt}}
\multiput(716.00,210.58)(4.251,0.498){75}{\rule{3.474pt}{0.120pt}}
\multiput(716.00,209.17)(321.789,39.000){2}{\rule{1.737pt}{0.400pt}}
\put(134.0,170.0){\rule[-0.200pt]{3.373pt}{0.400pt}}
\put(1045.0,249.0){\rule[-0.200pt]{94.915pt}{0.400pt}}
\sbox{\plotpoint}{\rule[-0.500pt]{1.000pt}{1.000pt}}%
\sbox{\plotpoint}{\rule[-0.200pt]{0.400pt}{0.400pt}}%
\put(1279,777){\makebox(0,0)[r]{Función raining-skyline ponderada}}
\sbox{\plotpoint}{\rule[-0.500pt]{1.000pt}{1.000pt}}%
\multiput(1299,777)(20.756,0.000){5}{\usebox{\plotpoint}}
\put(1399,777){\usebox{\plotpoint}}
\put(131,269){\usebox{\plotpoint}}
\multiput(131,269)(0.352,-20.753){3}{\usebox{\plotpoint}}
\put(135.26,210.00){\usebox{\plotpoint}}
\multiput(143,210)(2.574,-20.595){2}{\usebox{\plotpoint}}
\multiput(148,170)(20.322,4.222){28}{\usebox{\plotpoint}}
\multiput(716,288)(19.537,7.007){17}{\usebox{\plotpoint}}
\multiput(1045,406)(20.527,3.074){19}{\usebox{\plotpoint}}
\put(1439,465){\usebox{\plotpoint}}
\put(131,269){\makebox(0,0){$\circ$}}
\put(132,210){\makebox(0,0){$\circ$}}
\put(134,210){\makebox(0,0){$\circ$}}
\put(143,210){\makebox(0,0){$\circ$}}
\put(148,170){\makebox(0,0){$\circ$}}
\put(716,288){\makebox(0,0){$\circ$}}
\put(1045,406){\makebox(0,0){$\circ$}}
\put(1439,465){\makebox(0,0){$\circ$}}
\put(1349,777){\makebox(0,0){$\circ$}}
\sbox{\plotpoint}{\rule[-0.400pt]{0.800pt}{0.800pt}}%
\sbox{\plotpoint}{\rule[-0.200pt]{0.400pt}{0.400pt}}%
\put(1279,736){\makebox(0,0)[r]{Función raining-skyline ponderada por filas eliminadas}}
\sbox{\plotpoint}{\rule[-0.400pt]{0.800pt}{0.800pt}}%
\put(1299.0,736.0){\rule[-0.400pt]{24.090pt}{0.800pt}}
\put(131,308){\usebox{\plotpoint}}
\put(129.84,308){\rule{0.800pt}{14.213pt}}
\multiput(129.34,308.00)(1.000,29.500){2}{\rule{0.800pt}{7.107pt}}
\put(131.34,367){\rule{0.800pt}{9.395pt}}
\multiput(130.34,367.00)(2.000,19.500){2}{\rule{0.800pt}{4.698pt}}
\multiput(135.40,406.00)(0.516,2.441){11}{\rule{0.124pt}{3.756pt}}
\multiput(132.34,406.00)(9.000,32.205){2}{\rule{0.800pt}{1.878pt}}
\multiput(144.38,446.00)(0.560,12.850){3}{\rule{0.135pt}{12.840pt}}
\multiput(141.34,446.00)(5.000,52.350){2}{\rule{0.800pt}{6.420pt}}
\multiput(148.00,526.41)(4.865,0.502){111}{\rule{7.902pt}{0.121pt}}
\multiput(148.00,523.34)(551.600,59.000){2}{\rule{3.951pt}{0.800pt}}
\multiput(716.00,585.41)(9.022,0.506){31}{\rule{14.053pt}{0.122pt}}
\multiput(716.00,582.34)(299.833,19.000){2}{\rule{7.026pt}{0.800pt}}
\multiput(1045.00,604.41)(10.242,0.505){33}{\rule{15.960pt}{0.122pt}}
\multiput(1045.00,601.34)(360.874,20.000){2}{\rule{7.980pt}{0.800pt}}
\put(131,308){\makebox(0,0){$\ast$}}
\put(132,367){\makebox(0,0){$\ast$}}
\put(134,406){\makebox(0,0){$\ast$}}
\put(143,446){\makebox(0,0){$\ast$}}
\put(148,525){\makebox(0,0){$\ast$}}
\put(716,584){\makebox(0,0){$\ast$}}
\put(1045,603){\makebox(0,0){$\ast$}}
\put(1439,623){\makebox(0,0){$\ast$}}
\put(1349,736){\makebox(0,0){$\ast$}}
\sbox{\plotpoint}{\rule[-0.200pt]{0.400pt}{0.400pt}}%
\put(131.0,131.0){\rule[-0.200pt]{0.400pt}{175.375pt}}
\put(131.0,131.0){\rule[-0.200pt]{315.097pt}{0.400pt}}
\put(1439.0,131.0){\rule[-0.200pt]{0.400pt}{175.375pt}}
\put(131.0,859.0){\rule[-0.200pt]{315.097pt}{0.400pt}}
\end{picture}

\caption[short caption]{Relación de crecimiento de filas eliminadas por semillas usando 
la función de \emph{raining skyline} con y sin ponderar las casillas por el número de filas eliminadas.}
\label{fig:rainingskylinenoptres}
\end{figure} 

\subsection{Función híbrida}

Se ha creado una tercera función de evaluación que no es 
más que el producto de ambas funciones previamente definidas. 
No utiliza todos los pesos negativos sino que divide sobre el número 
de casillas cubiertas. 

\begin{displaymath}
    f(T) = \frac{rs \times (1 + \texttt{filas\_eliminadas})}{(1 + \texttt{cubiertos})}.
\end{displaymath} 

Las pruebas realizadas con colmenas de cien, quinientos y mil abejas no mostraron un cambio significativo 
en el comportamiento; por el contrario, los resultados de tres de las 5 pruebas 
resultaron en resultados pobres a comparación de previos resultados con cada 
función por lo que se descartó la función.

\section{Análisis de resultados}

Existen además de los resultados de la función de \textit{fitness} otros 
factores que modifican e influyen tanto positiva como negativamente el resultado 
de una ejecución del programa. A continuación se discuten algunos como 
el tamaño de la colmena y la búsqueda de semillas para el generador de números pseudoaleatorio.

\subsection{Tamaño de la colmena}

El tamaño de la colmena tiene una gran influencia en la ejecución y su tamaño
se mantuvo constante en los resultados anteriores: 50 abejas observadoras y 
50 abejas exploradoras para comenzar la colmena.

 Se puede observar con pocas semillas que si se incrementa el tamaño de la colmena, la calidad 
de las soluciones también lo hace. La mayor desventaja de aumentar el tamaño de 
la colonia es que el tiempo de ejecución por parte de la colmena también aumenta drásticamente;
una prueba realizada en la computadora que se describe en el \cref{sec:ambiente-fisico}, 
muestra que corriendo la función de evaluación \textit{filas entre pesos negativos} en una colmena de 
tamaño de 100 abejas, (50 exploradoras y 50 observadoras) tardan 9.96 segundos en 
terminar de crear un juego con seis filas removidas. Bajo las mismas condiciones pero 
con una colonia de tamaño mil, (500 exploradoras y 500 observadoras) las abejas 
tardan 4 minutos con 35.7 segundos, sin embargo el resultado del juego de prueba 
eliminó 28 filas.

Usando la función \emph{raining skyline} con una colmena de cien abejas, el número de 
filas que desaparece son cinco con sólo una ejecución que toma 8.76 segundos. 
Al elevar el número de abejas de nuevo a mil, la solución entregada es 1.3 veces 
mejor, con un resultado de 91 piezas jugadas y 26 filas removidas. Es de nuevo el 
tiempo de respuesta el que se eleva a 3 minutos con 34.48 segundos.

De las $22,000$ semillas que se corrieron para probar el comportamiento de las 
funciones de néctar, $15,000$ se corrieron con una colonia de 
tamaño cien, $6,000$ con una colonia de tamaño $200$ y $1,000$ con tamaño mil. 
El resultado de las colonias de mil sobre las de cien 
abejas obtuvo una mejor solución por un aproximado de 490\% filas más pero el 
tiempo de iteración sobre las semillas se incrementó en 26 veces el tiempo que llevó 
ejecutar las colmenas de cien abejas. Este resultado de tiempo es 
poco práctico para un juego de Tetris en tiempo real. 
 
 \begin{figure}[h]
% GNUPLOT: LaTeX picture
\setlength{\unitlength}{0.240900pt}
\ifx\plotpoint\undefined\newsavebox{\plotpoint}\fi
\sbox{\plotpoint}{\rule[-0.200pt]{0.400pt}{0.400pt}}%
\begin{picture}(1500,900)(0,0)
\sbox{\plotpoint}{\rule[-0.200pt]{0.400pt}{0.400pt}}%
\put(151.0,131.0){\rule[-0.200pt]{4.818pt}{0.400pt}}
\put(131,131){\makebox(0,0)[r]{$0$}}
\put(1419.0,131.0){\rule[-0.200pt]{4.818pt}{0.400pt}}
\put(151.0,859.0){\rule[-0.200pt]{4.818pt}{0.400pt}}
\put(131,859){\makebox(0,0)[r]{$100$}}
\put(1419.0,859.0){\rule[-0.200pt]{4.818pt}{0.400pt}}
\put(151.0,131.0){\rule[-0.200pt]{0.400pt}{4.818pt}}
\put(151,90){\makebox(0,0){$0$}}
\put(151.0,839.0){\rule[-0.200pt]{0.400pt}{4.818pt}}
\put(409.0,131.0){\rule[-0.200pt]{0.400pt}{4.818pt}}
\put(409,90){\makebox(0,0){$200$}}
\put(409.0,839.0){\rule[-0.200pt]{0.400pt}{4.818pt}}
\put(666.0,131.0){\rule[-0.200pt]{0.400pt}{4.818pt}}
\put(666,90){\makebox(0,0){$400$}}
\put(666.0,839.0){\rule[-0.200pt]{0.400pt}{4.818pt}}
\put(924.0,131.0){\rule[-0.200pt]{0.400pt}{4.818pt}}
\put(924,90){\makebox(0,0){$600$}}
\put(924.0,839.0){\rule[-0.200pt]{0.400pt}{4.818pt}}
\put(1181.0,131.0){\rule[-0.200pt]{0.400pt}{4.818pt}}
\put(1181,90){\makebox(0,0){$800$}}
\put(1181.0,839.0){\rule[-0.200pt]{0.400pt}{4.818pt}}
\put(1439.0,131.0){\rule[-0.200pt]{0.400pt}{4.818pt}}
\put(1439,90){\makebox(0,0){$1000$}}
\put(1439.0,839.0){\rule[-0.200pt]{0.400pt}{4.818pt}}
\put(151.0,131.0){\rule[-0.200pt]{0.400pt}{175.375pt}}
\put(151.0,131.0){\rule[-0.200pt]{310.279pt}{0.400pt}}
\put(1439.0,131.0){\rule[-0.200pt]{0.400pt}{175.375pt}}
\put(151.0,859.0){\rule[-0.200pt]{310.279pt}{0.400pt}}
\put(30,495){\rotatebox{90}{\makebox(0,0){Tiempo promedio de ejecución en segundos}}
}\put(795,29){\makebox(0,0){Tamaño de la colmena}}
\put(164,138){\usebox{\plotpoint}}
\multiput(164.00,138.58)(2.017,0.497){55}{\rule{1.700pt}{0.120pt}}
\multiput(164.00,137.17)(112.472,29.000){2}{\rule{0.850pt}{0.400pt}}
\multiput(280.00,167.58)(0.551,0.499){231}{\rule{0.541pt}{0.120pt}}
\multiput(280.00,166.17)(127.877,117.000){2}{\rule{0.271pt}{0.400pt}}
\multiput(409.58,284.00)(0.500,0.632){769}{\rule{0.120pt}{0.606pt}}
\multiput(408.17,284.00)(386.000,486.743){2}{\rule{0.400pt}{0.303pt}}
\multiput(795.00,772.58)(3.713,0.499){171}{\rule{3.061pt}{0.120pt}}
\multiput(795.00,771.17)(637.647,87.000){2}{\rule{1.530pt}{0.400pt}}
\put(151.0,131.0){\rule[-0.200pt]{0.400pt}{175.375pt}}
\put(151.0,131.0){\rule[-0.200pt]{310.279pt}{0.400pt}}
\put(1439.0,131.0){\rule[-0.200pt]{0.400pt}{175.375pt}}
\put(151.0,859.0){\rule[-0.200pt]{310.279pt}{0.400pt}}
\end{picture}

\caption[short caption]{Resultado de evaluar promedios de tiempos de ejecución 
con distintos tamaños de la colmena.}
\label{fig:tiempotamanio}
\end{figure} 

Por último, se debe mencionar que un mayor tiempo de ejecución no es sinónimo 
de un mal desempeño en una función; las mejores funciones deberán hacer 
que el tablero juegue más piezas, por lo que calcular la posición de las 
piezas siguientes llevan mayor tiempo de ejecución. Prueba de ello es la mejor solución 
encontrada durante la experimentación, se encontró en un tiempo aproximado de 7 minutos.

\subsection{Experimentación de semillas}

Para cada función de evaluación de probaron al menos $10,000$ semillas diferentes 
y se observó un comportamiento similar a la hora de experimentar con muchos 
valores; el número de pruebas que las distintas semillas contribuían a la 
búsqueda de un mejor tablero, se estabilizaba después de las primeras cien semillas. 

Independientemente de la función usada, el número de filas removidas es mejorada 
en el uso de las primeras 15 semillas, posteriormente hay un aumento de la 
función de costo muy reducido y se estabiliza. En el caso de la experimentación 
con las 10,000 semillas, ninguna función mejoró pasando la semilla $6,000$ y 
sólo una mejoró después de la $3,500$.

\subsection{Resultados de funciones}

El éxito de la heurística consiste en delegar toda responsabilidad a la definición 
de la función \textit{fitness} y debido a la naturaleza de las funciones generadoras 
de números aleatorios, es imposible conocer con certeza, como ya se ha discutido, 
si una función es estrictamente mejor que otra para una mayor cantidad de listas 
de piezas de entrada. Dicho lo anterior, se tomó una muestra para comparar los resultados 
de las dos funciones definidas en este capítulo. Lo que se busca con esta comparación 
es mostrar de manera informal una similitud de comportamiento con base a un conjunto 
de pruebas pequeño pero modificando los mismos parámetros para ambas funciones.

Para la muestra que se tomó se consideró una colmena de 300 abejas, un límite de 
iteración de cada fuente de 50, una distancia $\delta$ de 0.07 y los siguientes 
pesos: 

 \begin{displaymath} 
  \begin{array}{rcl}
    \texttt{PESO\_HORIZONTALIDAD} & = & 1 \\
    \texttt{PESO\_ATRAPADOS} & = & 3  \\
    \texttt{PESO\_CUBIERTOS} & = & 2  \\
    \texttt{PESO\_FILA\_REMOVIDA} & = & 1000  \\
    \texttt{PESO\_ALTURA} & = & 1.  
  \end{array}
\end{displaymath}

Los resultados son mostrados en la gráfica de la \cref{fig:dosfunciones} 
y se puede corroborar que el comportamiento de ambas funciones es muy similar. 

Para colmenas muy grandes, la función de división de pesos mostró una ligera 
mejoría tanto como en tiempo de respuesta así como en resultados obtenidos, sin 
embargo, la \emph{raining skyline} demostró ser más eficiente en cuanto a estabilidad 
de resultados se trata.

 \begin{figure}[H]
% GNUPLOT: LaTeX picture
\setlength{\unitlength}{0.240900pt}
\ifx\plotpoint\undefined\newsavebox{\plotpoint}\fi
\sbox{\plotpoint}{\rule[-0.200pt]{0.400pt}{0.400pt}}%
\begin{picture}(1500,900)(0,0)
\sbox{\plotpoint}{\rule[-0.200pt]{0.400pt}{0.400pt}}%
\put(131.0,131.0){\rule[-0.200pt]{4.818pt}{0.400pt}}
\put(111,131){\makebox(0,0)[r]{$0$}}
\put(1419.0,131.0){\rule[-0.200pt]{4.818pt}{0.400pt}}
\put(131.0,235.0){\rule[-0.200pt]{4.818pt}{0.400pt}}
\put(111,235){\makebox(0,0)[r]{$5$}}
\put(1419.0,235.0){\rule[-0.200pt]{4.818pt}{0.400pt}}
\put(131.0,339.0){\rule[-0.200pt]{4.818pt}{0.400pt}}
\put(111,339){\makebox(0,0)[r]{$10$}}
\put(1419.0,339.0){\rule[-0.200pt]{4.818pt}{0.400pt}}
\put(131.0,443.0){\rule[-0.200pt]{4.818pt}{0.400pt}}
\put(111,443){\makebox(0,0)[r]{$15$}}
\put(1419.0,443.0){\rule[-0.200pt]{4.818pt}{0.400pt}}
\put(131.0,547.0){\rule[-0.200pt]{4.818pt}{0.400pt}}
\put(111,547){\makebox(0,0)[r]{$20$}}
\put(1419.0,547.0){\rule[-0.200pt]{4.818pt}{0.400pt}}
\put(131.0,651.0){\rule[-0.200pt]{4.818pt}{0.400pt}}
\put(111,651){\makebox(0,0)[r]{$25$}}
\put(1419.0,651.0){\rule[-0.200pt]{4.818pt}{0.400pt}}
\put(131.0,755.0){\rule[-0.200pt]{4.818pt}{0.400pt}}
\put(111,755){\makebox(0,0)[r]{$30$}}
\put(1419.0,755.0){\rule[-0.200pt]{4.818pt}{0.400pt}}
\put(131.0,859.0){\rule[-0.200pt]{4.818pt}{0.400pt}}
\put(111,859){\makebox(0,0)[r]{$35$}}
\put(1419.0,859.0){\rule[-0.200pt]{4.818pt}{0.400pt}}
\put(131.0,131.0){\rule[-0.200pt]{0.400pt}{4.818pt}}
\put(131,90){\makebox(0,0){$0$}}
\put(131.0,839.0){\rule[-0.200pt]{0.400pt}{4.818pt}}
\put(393.0,131.0){\rule[-0.200pt]{0.400pt}{4.818pt}}
\put(393,90){\makebox(0,0){$200$}}
\put(393.0,839.0){\rule[-0.200pt]{0.400pt}{4.818pt}}
\put(654.0,131.0){\rule[-0.200pt]{0.400pt}{4.818pt}}
\put(654,90){\makebox(0,0){$400$}}
\put(654.0,839.0){\rule[-0.200pt]{0.400pt}{4.818pt}}
\put(916.0,131.0){\rule[-0.200pt]{0.400pt}{4.818pt}}
\put(916,90){\makebox(0,0){$600$}}
\put(916.0,839.0){\rule[-0.200pt]{0.400pt}{4.818pt}}
\put(1177.0,131.0){\rule[-0.200pt]{0.400pt}{4.818pt}}
\put(1177,90){\makebox(0,0){$800$}}
\put(1177.0,839.0){\rule[-0.200pt]{0.400pt}{4.818pt}}
\put(1439.0,131.0){\rule[-0.200pt]{0.400pt}{4.818pt}}
\put(1439,90){\makebox(0,0){$1000$}}
\put(1439.0,839.0){\rule[-0.200pt]{0.400pt}{4.818pt}}
\put(131.0,131.0){\rule[-0.200pt]{0.400pt}{175.375pt}}
\put(131.0,131.0){\rule[-0.200pt]{315.097pt}{0.400pt}}
\put(1439.0,131.0){\rule[-0.200pt]{0.400pt}{175.375pt}}
\put(131.0,859.0){\rule[-0.200pt]{315.097pt}{0.400pt}}
\put(30,495){\rotatebox{90}{\makebox(0,0){Filas eliminadas}}
}\put(785,29){\makebox(0,0){Número de pruebas}}
\put(1279,818){\makebox(0,0)[r]{Función raining-skyline}}
\put(1299.0,818.0){\rule[-0.200pt]{24.090pt}{0.400pt}}
\put(131,235){\usebox{\plotpoint}}
\multiput(131.00,235.58)(0.524,0.500){621}{\rule{0.519pt}{0.120pt}}
\multiput(131.00,234.17)(325.922,312.000){2}{\rule{0.260pt}{0.400pt}}
\multiput(785.00,547.58)(1.976,0.499){163}{\rule{1.676pt}{0.120pt}}
\multiput(785.00,546.17)(323.522,83.000){2}{\rule{0.838pt}{0.400pt}}
\multiput(1112.00,630.58)(3.921,0.498){81}{\rule{3.214pt}{0.120pt}}
\multiput(1112.00,629.17)(320.329,42.000){2}{\rule{1.607pt}{0.400pt}}
\put(458.0,547.0){\rule[-0.200pt]{78.774pt}{0.400pt}}
\sbox{\plotpoint}{\rule[-0.600pt]{1.200pt}{1.200pt}}%
\sbox{\plotpoint}{\rule[-0.200pt]{0.400pt}{0.400pt}}%
\put(1279,777){\makebox(0,0)[r]{Función de pesos pos/neg}}
\sbox{\plotpoint}{\rule[-0.600pt]{1.200pt}{1.200pt}}%
\put(1299.0,777.0){\rule[-0.600pt]{24.090pt}{1.200pt}}
\put(131,256){\usebox{\plotpoint}}
\multiput(131.00,258.24)(0.984,0.500){322}{\rule{2.664pt}{0.120pt}}
\multiput(131.00,253.51)(321.471,166.000){2}{\rule{1.332pt}{1.200pt}}
\multiput(458.00,424.24)(0.873,0.500){364}{\rule{2.398pt}{0.120pt}}
\multiput(458.00,419.51)(322.022,187.000){2}{\rule{1.199pt}{1.200pt}}
\multiput(1112.00,611.24)(1.574,0.500){198}{\rule{4.073pt}{0.120pt}}
\multiput(1112.00,606.51)(318.546,104.000){2}{\rule{2.037pt}{1.200pt}}
\put(131,256){\makebox(0,0){$\blacksquare$}}
\put(458,422){\makebox(0,0){$\blacksquare$}}
\put(785,609){\makebox(0,0){$\blacksquare$}}
\put(1112,609){\makebox(0,0){$\blacksquare$}}
\put(1439,713){\makebox(0,0){$\blacksquare$}}
\put(1349,777){\makebox(0,0){$\blacksquare$}}
\put(785.0,609.0){\rule[-0.600pt]{78.774pt}{1.200pt}}
\sbox{\plotpoint}{\rule[-0.200pt]{0.400pt}{0.400pt}}%
\put(131.0,131.0){\rule[-0.200pt]{0.400pt}{175.375pt}}
\put(131.0,131.0){\rule[-0.200pt]{315.097pt}{0.400pt}}
\put(1439.0,131.0){\rule[-0.200pt]{0.400pt}{175.375pt}}
\put(131.0,859.0){\rule[-0.200pt]{315.097pt}{0.400pt}}
\end{picture}

\caption[short caption]{Comparación de las funciones propuestas sobre una muestra 
y valores específicos.}
\label{fig:dosfunciones}
\end{figure} 


\section{Conclusión de la evaluación}

La meta fue superada rápidamente conforme el refinamiento de las funciones 
de costo. Se consiguieron muestras mayores a las propuestas por el método 
de juego \textit{40 lines}. 

Los resultados obtenidos superaron las expectativas que se propusieron al inicio 
del diseño y las pruebas. Dos muestras de la experimentación se pueden encontrar 
en las direcciones \url{https://youtu.be/F-Nxjvu8fPA} y 
\url{https://youtu.be/rjpVh7kaREY}. La primera muestra toma la función de 
\emph{raining skyline} con 500 abejas, esta muestra supera la primera meta propuesta 
como objetivo de la experimentación. La segunda muestra toma la función de pesos 
y una colmena de tamaño 1000. Esta muestra supera las 70 filas eliminadas y en el 
video se está documentando el historial de movimientos del resultado 
con casi nueve minutos de juego.

%%% Local Variables:
%%% mode: latex
%%% TeX-master: "../main"
%%% End:
