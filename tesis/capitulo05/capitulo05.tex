\chapter{Tecnologías utilizadas}

Aunque el concepto de complejidad de un algoritmo, clases de complejidad y
eficiencia mantienen un estado invariante al entorno de desarrollo de
los programas, el análisis de la experimentación está inevitablemente
comprometido por el desempeño del hardware y software que se usa. Para
darle sentido a los resultados, es muy importante conocer las herramientas 
usadas, así como el ambiente en el que los experimentos fueron realizados.

\section{Python 3.5}

En la década de los ochenta, en el centro Wiskunde de informática, en Ámsterdam,
Guido van Rossum trabajaba como desarrollador de un lenguaje de programación
llamado \textit{ABC} cuyo propósito era ser una herramienta de desarrollo
fácil de aprender para personas no acostumbradas a programar. Rossum vio la
necesidad de crear un lenguaje de \textit{scripting}\footnote{En
algunos textos referidos como \textit{Guiones del intérprete de comandos}, son
programas en las que sus órdenes son ejecutadas de manera
secuencial~\cite{interprete-comandos}.} sobre su proyecto en el lenguaje ABC,
por lo que creó una máquina virtual, un programa \emph{parser} y otro de ejecución de
comandos; agregó una sintaxis básica, usó indentación para definir bloques
y creó un pequeño conjunto de tipos de datos. Así nació
Python~\cite{python-interview}.

Python es un lenguaje de programación de propósito general, Turing completo,
dinámicamente tipificado y con influencia de múltiples paradigmas; incluidos
pero no limitado a orientación a objetos, funcional e imperativo~\cite{python}.
Una de las principales características por las que Python es reconocido, es por
su manera de definir bloques de código: la indentación de un programa en Python
es fundamental para la semántica de un programa.

Python se encuentra desde el 2003 como uno de los lenguajes de programación
más populares de acuerdo al índice de clasificación de la comunidad programadora
TIOBE~\cite{tiobe}. Cientos de compañías como Google, Yahoo, Disney Animation,
NASA, IBM, usan Python como lenguaje de desarrollo para resolver
problemas en diversas áreas~\cite{python-usage}.

Actualmente existen muchas versiones de Python; las más usadas son 
las versiones 2.7 y 3.5 que en conjunto con el administrador de
paquetes\footnote{Un administrador de paquetes es un programa que tiene como
propósito la instalación, actualización y configuración de paquetes de software.}
PIP, proveen de cientos de bibliotecas listas para usar e implementar soluciones
a problemáticas de diversa índole~\cite{python}.

Otra de las ventajas más grande que posee Python sobre otros lenguajes de alto
nivel, es su ambiente de desarrollo
y ejecución; Python posee como herramienta la creación de sus propios ambientes
virtuales con sus propio sistema de archivos aislados e independientes al
sistema local. La ventaja principal de mantener un ambiente de desarrollo
independiente, es la instalación y manejos de paquetes en un sistema donde
aquellos que poseen otros ambientes virtuales o el mismo ambiente
local no interfiera ni genere conflictos por paquetes previamente instalados o
por versiones diferentes~\cite{venv}. %python3 -m venv ./.env

En los últimos años Python en su desarrollo se ha orientado a mejorar la
eficiencia de tiempo de respuesta de sus llamadas a sistema; la forma en la que
Python mejora su velocidad aún siendo un lenguaje que típicamente usa un
intérprete, es generando archivos con extensión \texttt{.pyc}, que son archivos
en lenguaje máquina, los cuales contienen funciones e instrucciones previamente
ejecutadas y optimizadas para su posterior uso repetitivo~\cite{python}. De
cualquier manera, Python sigue teniendo algunos problemas para escalar a
sistemas grandes y complejos.


\subsection{pygame}

Una de las razones por las que Python se hizo altamente reconocido, fue por la
facilidad de hacer uso de bibliotecas externas al núcleo del lenguaje. Existen
una gran cantidad de bibliotecas muy conocidas y usadas mundialmente,
ya sea por la facilidad que proveen o por lo conveniente que resultan sus
soluciones. Un ejemplo es la
biblioteca Numpy que posee un conjunto de funciones de uso científico ampliamente
reconocido por la comunidad investigadora, por ser de gran utilidad en numerosos 
proyectos~\cite{numpy}.

La biblioteca pygame es una biblioteca de Python, de código abierto que sirve para realizar 
aplicaciones multimedia como animaciones o videojuegos; existen 
alternativas ya programadas que tiene como propósito ejecutar videojuegos 
publicados originalmente para la consola \textit{Nintendo Entertainment System} 
o \textit{NES}, sin embargo, las opciones consideradas presentan la inconveniencia de no poder 
modificar el tablero a voluntad de manera directa y simplificada.~\cite{gym-tetris,nespy}.

Si bien el propósito de las bibliotecas pygame puede ser la creación de 
videojuegos donde la optimización de gráficos y ejecución requiere el mayor 
cuidado, sólo se usará como una herramienta de visualización de resultados debido 
a la naturaleza de la heurística de colonia de abejas artificiales~\cite{pygame}.



\section{Git}

Durante el desarrollo de cualquier proyecto de software existen muchas herramientas útiles para
mejorar y hacer más eficiente el flujo de trabajo. Un ejemplo de software útil son
los controladores de versiones, también llamados \textit{versionadores}.
Los controladores de versiones son sistemas que
mantienen un registro (en un lugar llamado repositorio) de todos los cambios
realizados a un conjunto de archivos rastreables, produciendo la oportunidad de
observar un historial de modificaciones realizadas en un periodo de
tiempo~\cite{git-about}. Existen dos categorías principales de controlador de
versiones: los versionadores de repositorios centralizados que son aquellos en los que los
archivos que están siendo rastreados se encuentran en un repositorio central,
único, que todos los desarrolladores modifican; y los versionadores de repositorios
distribuidos que mantienen muchas copias del código en distintas computadoras e
implementan un sistema de comunicación de cambios mediante un historial. De este
último tipo es uno de los versionadores más usados: Git~\cite{scopatz2015effective}.

Git fue desarrollado en el 2005 por el mismo creador de uno de los proyectos más
grandes de software libre que existe; el proyecto Linux necesitaba de un controlador de
versiones debido a su particularidad de ser un sistema operativo que es
modificado por miles de personas alrededor del mundo. Linus Torvalds
consideró necesario la creación de Git como herramienta rápida, segura y que
soportara desarrollo de código de manera no lineal, para mantener una buena
comunicación de desarrollo del proyecto Linux~\cite{git-about2,git-commit}.

Si bien el uso básico de Git es sencillo, dominarlo puede llegar a ser un
trabajo de años de experiencia. Un programador sin experiencia en la herramienta
puede comenzar a realizar seguimiento y registro de cambios en el sistema de
archivos con un par de comandos (\texttt{git init}, \texttt{git add -A},
\texttt{git commit} en particular).

Los cambios son guardados dentro de un árbol de registros donde cada rama 
del árbol pertenecerá a una secuencia de historias. 
Cada cambio agregado y almacenado genera una huella digital única llamada \textit{hash}. 
Este \emph{hash} único es el identificador del espacio
temporal del historial donde se almacena la información $I$ en el tiempo $T$~\cite{git-about}.

Una de las conveniencias de usar Git como herramienta para proyectos de desarrollo 
de software es poder regresar a algún punto en una versión del historial. Es
conveniente conocer el \emph{hash} con el comando \texttt{git log} y familiarizarse con 
la orden \texttt{git revert}.

Es importante aclarar que el uso de Git va más allá de mantener un registro 
y poder mover el estado actual del proyecto por uno de sus estados 
almacenados; Git posee características que benefician al trabajo en equipo,
depuración de errores, pruebas y mantenimiento, desarrollo remoto y varias
características que lo han hecho uno de los programas con mayor número de usuarios y 
de traducciones: Git existe en doce idiomas y más de seis traducciones parciales.

Durante el desarrollo de este proyecto se usará Git de manera constante
para llevar un registro de todos los cambios hechos en el código de la
implementación de la heurística.

\section{Ambiente físico}
\label{sec:ambiente-fisico}

Invariablemente los resultados obtenidos pueden ser muy diferentes dependiendo 
el equipo en el que se prueben así como su sistema operativo y su versión correspondiente. 
El hardware es mejorado con notoriedad cada poco tiempo de acuerdo a las observaciones 
como las de Mark Kryder o la reconocida ley de Moore, por lo que no es de extrañarse 
que con el paso de tiempo los resultados parecieran mejorar. El software también 
sufre de mejoras; algoritmos de funciones de los sistemas operativos son actualizados 
constantemente para garantizar el mayor rendimiento posible. 

El desarrollo, pruebas y análisis fueron realizadas en una computadora 
ASUS, ZenBook Pro 14 modelo UX450FDX. La velocidad de comunicación de 
la memoria RAM, 8 GB DDR4 Synchronous es de 2,400 MHz (a 0.4 ns). Los 
tres niveles de cache (L1, L2 y L3) poseen respectivamente 256 KB, 1 MB y 6 MB.
El procesador del ambiente de ejecución es de la marca Intel(R) Core(TM), 
modelo i5-8265U, x86\_64, 8 núcleos de dos hilos cada uno y corre a una velocidad 
promedio máxima de 3.9 GHz. 

El sistema operativo instalado en la computadora es una distribución de GNU/Linux 
Debian, con una versión actualizada del kernel 4.19.0-0.bpo.4-amd64, que corre 
sobre la versión de la arquitectura de 64 bits. La versión del lenguaje de programación usada 
también es relevante debido a que los lenguajes de programación son constantemente 
modificados y mejorados. El ambiente de desarrollo que se usa es la versión 
de Python 3.5.3.

%%% Local Variables:
%%% mode: latex
%%% TeX-master: "../main"
%%% End:
