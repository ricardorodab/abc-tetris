\begin{figure}[H]
\centering
% 4 es Sq
% 1 es I
% 2 es L(Lg) inversa y 6 es L(Rg)
% 3 es S(Rs) y Z es 7(Ls)
% 5 es T 
% 1, 6, 2, 4, 1, 5, 7, 3, 1
% I, Rg, Lg, Sq, I, T, Ls, Rs, I
\begin{tikzpicture}[omino/nodes/.style={shape=rectangle, rounded corners, inner sep=+0pt, minimum size=1cm-2\pgflinewidth}]
	\path [omino={at={-2,0}, rotate=0}][tetris=1];
	\path [omino={at={4,0}, rotate=0}][tetris=1];
    \path [omino={at={2,0}, rotate=0}][tetris=4];
	\path [omino={at={1,0}, rotate=0}][tetris=2];	
	\path [omino={at={0,0}, rotate=0}][tetris=6];
	\path [omino={at={5,0}, rotate=0}][tetris=5];
	\path [omino={at={7,1}, rotate=0}][tetris=7];
	\path [omino={at={-2,4}, rotate=0}][tetris=3];
	\path [omino={at={0,3}, rotate=0}][tetris=1];
%    \path [omino/at=0:1] [tetris=2];
%    \path [omino/at=0:2] [tetris=3];
%    \path [omino={at=0:3, rotate=-90, x mirror}][tetris=5];
%    \path [omino={at={5,1}, rotate=-90}][tetris=3];
%    \path [omino={at={5,2}, rotate=-90}][tetris=2];
%    \path [omino={at={4,2}, x mirror}][tetris=5];
%    \path [omino={at={1,3}}][tetris=4];
\end{tikzpicture}
\caption{Una mala función de costo deja muchas casillas atrapadas y cubiertas.} \label{fig:mal-juego}
\end{figure}