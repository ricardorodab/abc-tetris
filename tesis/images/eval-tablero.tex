\begin{figure}[H]
\centering
% 4 es Sq
% 1 es I
% 2 es L(Lg) inversa y 6 es L(Rg)
% 3 es S(Rs) y Z es 7(Ls)
% 5 es T 
% 1, 1, 4, 4, 4, 4
% I, I, Sq, Sq, Sq, Sq
\begin{tikzpicture}[omino/nodes/.style={shape=rectangle, rounded corners, inner sep=+0pt, minimum size=1cm-2\pgflinewidth}]
	\path [omino={at={9,2}, rotate=180}][tetris=2];
	\path [omino={at={2,0}, rotate=0}][tetris=4];
	\path [omino={at={7,0}, rotate=90}][tetris=1];
	\path [omino={at={8,1}, rotate=90}][tetris=1];	
	\path [omino={at={0	,0}, rotate=0}][tetris=4];
	\path [omino={at={3,2}, rotate=-90}][tetris=5];
\end{tikzpicture}
\caption{Tablero que toma las piezas de la \cref{fig:eval-tablero} y la función de filas entre pesos negativos.} \label{fig:eval-tablero-dos}
\end{figure}