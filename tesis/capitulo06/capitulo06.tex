\chapter{Análisis y diseño de la implementación}

Un buen diseño de software 
apegado a buenas prácticas de programación contribuyen a la obtención 
de resultados de calidad, menos errores durante la implementación y un 
entendimiento más sencillo para terceros. Durante este capítulo, se analizan  
las decisiones de la implementación y el motivo del diseño. Todas las clases y 
funciones discutidas a continuación se encuentran en el 
\cref{apendice:code} y en la liga 
\url{https://github.com/ricardorodab/abc-tetris}.

\section{Análisis del sistema}

Para encontrar soluciones a la colocación de las piezas de Tetris, primero se tiene que 
partir del poder acceder a toda la información del juego para el análisis de los desenlaces. 
En el \cref{subsec:formalizacion} se realiza la formalización del juego y con ella las 
reglas a seguir durante la implementación:

\begin{enumerate}
\item Se debe diseñar un tablero que almacene la información del conjunto de 
piezas.

\item Se debe tener piezas que posean un tipo específico, posición y orientación. 

\item Se debe programar un modelo de rotación específico para las piezas de 
acuerdo a su tipo.

\item El tablero, piezas y modelo de rotación deben estar regidos en todo 
momento por las reglas del juego.
\end{enumerate} 

El objeto que contenga los entes principales del juego, deben también tener 
métodos de obtención de información relevantes a la heurística como datos que 
sirvan para generar una función de costo y métodos para forzar ciertos 
comportamientos indicados por un objeto externo (como lo es la misma heurística 
en este caso) para pasar de un estado del juego a un estado siguiente.

Una regla adicional necesaria para la utilización de cualquier conjunto de reglas 
a optimizar es que todo objeto que reciba la heurística debe ser ``clonable'', 
esto quiere decir que exista un método o función la cual regrese un objeto idéntico 
al que lo llama y que la heurística tenga la libertad de modificar sin afectar el 
estado del objeto clonado original.

Del lado de la heurística el elemento más importante a considerar es el 
comportamiento de la colmena como un conjunto de funciones que realizan 
las abejas que habitan en ella en un periodo de tiempo. Cada abeja 
creada dentro de la colmena tiene la tarea de explorar y trabajar o 
de observar. La necesidad de dividir a las abejas de esta manera se puede entender 
en el artículo \cite{karaboga2008performance} donde el autor de la heurística, 
Karaboga, realiza pruebas que concluyen en que tener el 50\% de abejas observadoras 
incrementa el néctar de sus soluciones. 

La ejecución del código ocurre de forma modulada, equivalente al modelo de 
\textit{pipelines} usado por Henry Ford. Durante una iteración de la colmena se 
deben realizar las siguientes acciones: 

\begin{enumerate}
\item Si la colmena no tiene abejas, inicializarlas.

\item Mandar a llamar al método que libera a las exploradoras y las transforma en 
trabajadoras asociadas a una fuente.

\item Mandar a llamar a las abejas trabajadoras para que recolecten néctar de su 
fuente y generen información para hacer el \textit{waggle-dance}.

\item Hacer que las abejas observadoras escojan una fuente y ejecutar las 
funciones implementadas para la observación de cada fuente vecina. 
 
\end{enumerate}

El problema específico a resolver no debe afectar el comportamiento de la 
heurística por lo que se puede aprovechar las propiedades de lenguajes en el 
paradigma funcional que ofrece 
Python y asignar funciones como parámetros y atributos para que la ejecución de las abejas 
sea independiente al problema y sean los atributos de cada abeja los 
llamados para evaluar las fuentes.

Un último paso para poder operar con la heurística y el juego es el conjunto 
de métodos que realizan la comunicación entre ambos; los resultados se deben unir de alguna 
manera ya que el resultado de cada iteración es la entrada de la 
siguiente. Las funciones que las abejas esperan y necesitan para realizar su 
trabajo dentro de la heurística también es necesario comunicarlo a la hora 
de instanciar la colmena.

Una buena práctica es que los parámetros de experimentación no sean modificados 
directamente en el código sino que se deben encontrar como entrada, en algún archivo 
de configuración para su fácil modificación y preservar así un buen diseño. Es 
conveniente implementar métodos asociados al manejo del archivo de entrada así como un objeto 
que contenga siempre estos parámetros desde el principio de la ejecución del programa.

\subsection{Abejas observadoras}

Las abejas observadoras tienen un nivel de dificultad un poco mayor de diseño e 
implementación debido a la naturaleza de las acciones que toman. Una abeja 
observadora primero le es asignada una fuente $i$ de las ya trabajadas con 
probabilidad 

\begin{displaymath}
  P_{i} = \frac{F(\theta_{i})}{\sum_{k=1}^{n} F(\theta_{k})}.
\end{displaymath}

Se debe considerar el caso donde después de recorrer todas las fuentes, la 
abeja observadora se queda sin alguna asignación de fuente. 
Una abeja observadora no se debe quedar 
sin fuente por lo que de existir un abeja en este caso se debe volver a iterar 
sobre todas las fuentes hasta que le sea asignada. Una solución para evitar este 
problema del no determinismo a la hora de asignar parejas, es duplicar la 
probabilidad de asignación de todas las soluciones por cada iteración para 
evitar que se recorra demasiada veces la lista de fuentes. 

Un problema aún mayor es la localización de fuentes vecinas. En el caso particular 
de Tetris, se necesita alguna manera de conseguir una fuente vecina a un punto 
del tablero. La definición de fuente vecina es 

\begin{displaymath}
  \theta_{i}(c+1) = \theta_{i}(c) \pm \phi_{i}(c)
\end{displaymath}

\noindent
donde $\theta_{i}(c)$ es la solución $i$-ésima. El primer enfoque resolutivo de este 
problema fue el jugar movimientos aleatorios posteriores a una fuente, 
sin embargo dichos movimientos serían correspondientes a un punto futuro de la partida 
por lo que no se considera una fuente vecina sino la fuente $i$ explotada como 
lo haría una abeja trabajadora. 
Una forma en la que $\phi_{i}(c)$ es encontrada es mantener un registro de 
movimientos que se jugaron para llegar a $\theta_{i}(c)$ y poder cambiar algunos 
movimientos previos con ayuda de un \textit{historial}.

El historial no es más que una lista de movimientos $L_{\theta}$ que se han hecho hasta un 
tiempo $T$. Es posible eliminar un número aleatorio $\delta_{1}$ de movimientos 
para posteriormente agregar en la lista otro número de movimientos aleatorios $\delta_{2}$ hasta 
llegar a $T$ con una lista de movimientos $L_{\phi}$. Los tiempos $T$ fueron seleccionados 
como los puntos en 
el que las piezas son fijadas, esto debido a que ese comportamiento es una invariante 
para todas las piezas del juego.

\begin{figure}[H]
\centering
\begin{tikzpicture}[
    node distance = 21mm and 7mm,
    box/.style = {draw, minimum size=8mm, inner sep=0pt, outer sep=0pt, anchor=center},
      pin edge = {Straight Barb-, shorten <=1mm,semithick}]
        \node (h2) [box] {Izq};
		\node (h1) [box, left=of h2] {Der};
		\node (h3) [box, right=of h2]    {Gir};
		\node (h4) [box, right=of h3]    {Cae};
		\node (q_dots) [draw=none, right=of h4] {$\cdots$};
		\node (hk) [box, right=of q_dots] {$Mov_{m}$};
		
		\node (h5) [box, below=of h4] {Izq};
		\node (h32) [draw=none, left=of h5] {};
		\node (q_dots_two) [draw=none, right=of h5] {$\cdots$};
		\node (hl) [box, below=of hk] {$Mov_{n}$};
		
		
		\draw[dashed,-Straight Barb, shorten <=1mm, shorten >=1mm]
    (h3.south) edge [bend right]   (h5.north);	
		
		\draw[-Straight Barb, transform canvas={yshift= 0mm}]
    		(h1) edge   (h2);
    	\draw[-Straight Barb, transform canvas={yshift= 0mm}]
    		(h2) edge   (h3);
    	\draw[-Straight Barb, transform canvas={yshift= 0mm}]
    		(h3) edge   (h4);
    	\draw[-Straight Barb, transform canvas={yshift= 0mm}]
    		(h4) edge   (q_dots);    		
    	\draw[-Straight Barb, transform canvas={yshift= 0mm}]
    		(h5) edge   (q_dots_two);
    	\draw[-Straight Barb, transform canvas={yshift= 0mm}]
    		(q_dots_two) edge   (hl);
    	\draw[-Straight Barb, transform canvas={yshift= 0mm}]
    		(q_dots) edge   (hk);
    	\draw [decorate,decoration={brace,amplitude=6pt,raise=5ex}]
  (h1) -- (hk) node[midway,yshift=4em]{$\theta_{i}$};
  
 		 \draw [decorate,decoration={brace,amplitude=7pt,mirror,raise=5ex}]
  (h32) -- (hl) node[midway,yshift=-4em]{$\phi_{i}$};
    \end{tikzpicture}
\caption{Historial de movimientos dentro de un juego de Tetris.} \label{fig:listmov}
\end{figure}

\subsection{Función de costo}

La función de costo o también conocida como función \textit{fitness} es la encargada 
de calificar una solución propuesta por la colmena para su posterior uso. 
Si una función de costo es de baja calidad, los resultados serán de igual forma 
de baja calidad. Una buena función de costo deberá premiar con una mejor 
calificación a los tableros que resulten con una mejor estrategia de juego y 
deberá desanimar aquellos tableros que se consideren resultados de malas decisiones. Es 
necesario que por cada modo de juegos o solicitud a resolver exista una función 
de costo enfocada a resolver dicha solicitud como en el caso de Tetris y la 
heurística de abejas artificiales, la cual usa una función de costo llamada 
``néctar'' que es de tipo \textit{higher is better} (entre más grande el 
resultado, mejor solución).

Adicionalmente a la función para conseguir el néctar, la colmena necesita 
otras tres funciones para su correcto funcionamiento y ofrece una cuarta 
función completamente opcional para un comportamiento positerativo: %Es válido

\begin{itemize}
\item\textbf{Función para buscar fuente:} Esta función es la que usarán 
las abejas exploradoras para entregar una fuente que habrá que explotar.

\item\textbf{Función para explotar fuente:} Esta función es la que usarán 
las abejas trabajadoras para seguir explotando una fuente y modificar su estado.

\item\textbf{Función de observación:} Esta función es la que usarán 
las abejas observadoras y, para la instancia de Tetris específica, es la 
función que eliminará movimientos en el historial para crear una nueva 
combinación de movimientos jugados.

\item\textbf{Función opcional de término:} Esta función realiza una acción adicional 
sobre las fuentes al finalizar la iteración de la colmena, en el caso de Tetris 
limpia las filas si es que se deben limpiar.
\end{itemize}

\section{Diseño del sistema}

Para lograr el objetivo de crear un sistema que resuelva hacer operaciones 
con el problema de Tetris, operar con una heurística para encontrar soluciones y 
comunicarse entra ambos con una lógica sencilla pero manteniendo buenas prácticas de 
programación, se dividió en tres componentes principales para poder 
lograr apegarse al principio de orientación a objetos: 
la heurística ABC, el juegos de Tetris y la comunicación Abejas-Tetris.

\begin{figure}[H]
\centering
\begin{tikzpicture}
\umlsimpleclass[x=-6, y=1]{\_\_main\_\_.py}
\begin{umlpackage}[x=-6,y=0.7,scale=0.1, name=union]{}
\end{umlpackage}


\begin{umlpackage}[x=0 ,y=0]{test}
\dots
\end{umlpackage}

\begin{umlpackage}[x=0 ,y=0]{abejas-tetris}

\begin{umlpackage}[x=-3 ,y=-3]{abc}
\dots
\end{umlpackage}

\begin{umlpackage}[x=3 ,y=3]{tetris}
\dots
\end{umlpackage}

\umlclass[x=0, y=0]{Abejas\_Tetris}{
$\qquad\quad\dots$
}{
$\qquad\quad\dots$ 
}

\end{umlpackage}
\umlimport[geometry=|-,anchor1=0, anchor2=west, name=import]{abc}{test}
\umlimport[geometry=|-, name=import]{tetris}{test}
\umlimport[geometry=|-,anchor1=north, anchor2=90, name=import]{test}{union}
\end{tikzpicture}
\caption{Estructura básica del sistema.} \label{fig:uml-estructura}
\end{figure}

\subsection{Orientación a Objetos}

La decisión de realizar un diseño que siga los principios de la Orientación a 
Objetos (u OO) tiene como sustento la abstracción necesaria de los entes participantes 
de la heurística, que responderán a estímulos o mensajes recibidos mediante 
la ejecución de métodos o funciones del juego de Tetris. 

Para un programador acostumbrado a la OO, el modelo propuesto por Dervis 
Karaboga~\cite{karaboga2005idea} donde enuncia un conjunto de individuos 
participantes como son las abejas, colmena y fuentes, posee una relación bastante 
fácil de traducir a entes que tienen estructura, comportamiento e identidad (que son 
las componentes principales que todo objeto tiene en su 
descripción~\cite{gaona2007introduccion}) y que comparten un propósito definido 
en la heurística sin importar su comportamiento unitario. 

La experiencia personal como estudiante de la carrera de Ciencias de la Computación 
en la Facultad de Ciencias, suma un conjunto de prácticas donde la abstracción 
a la hora de realizar la toma de decisión sobre las clases, atributos y métodos 
así como el lenguaje de programación, paradigmas y tecnologías son refinada con 
proyectos, clases y prácticas desde el primer semestre hasta el último de la 
estancia como alumno. 

Por último, el paradigma orientado a objetos tiene como una de sus 
características principales más atractivas lo intuitivo de la descripción 
de objetos reales, con atributos que a su vez pueden ser objetos con sus 
propios atributos compuestos previamente definidos~\cite{Lewis:2007:JSS:1554921}.  
Por parte del juego de Tetris esta característica es muy atractiva debido a que 
el tablero está constituido por un cojunto de entes (como son las casillas, piezas, 
puntos) que habrá que definir a su vez y que en conjunto posee un 
comportamiento que dependerá del estado unitario así como la comunicación de 
todos los objetos dentro del juego. 

\subsection{Diseño de Tetris}

Una partida de Tetris lleva varios agentes involucrados como lo son 
el tablero, que puede estar vacío o formado por piezas; y las piezas 
que están formadas por cuatro casillas y posee un identificador llamado 
\textit{tipo}, que es alguna constante para indicar cuál de las siete posibles 
piezas es. Cada casilla que conforma la pieza posee un estado de movimiento 
que puede ser fijo o movible y un punto de la forma $(x,y)$ que indica 
su posición en el tablero. 

Se definieron cinco clases y dos enumeraciones para construir una lógica 
robusta y modular. La primera clase y la más sencilla es la clase \texttt{Punto}, 
conformado por dos números enteros, $x$ y $y$. Cada objeto de la clase 
\texttt{Casilla} posee un objeto punto y un atributo llamado \texttt{\_fija} que 
es una bandera para conocer el estado actual de esa casilla; si la bandera 
se encuentra en \texttt{True} significa que la casilla es parte de una 
pieza que se está jugando actualmente.

La clase \texttt{Pieza} por su parte posee un atributo del tipo \texttt{Tipo}, que 
es una enumeración con siete posibles valores que sirve para identificar a los tetrominós, 
una orientación de la forma \linebreak $x = (90 \times k)$ $\mod$ $360$, una posición que es un punto 
de la casilla principal de la pieza y una lista de cuatro elementos que contiene 
a las casillas. 

El tablero del juego posee referencias a las casillas de las piezas en una matriz 
bidimensional de casillas, una pieza actual y una pieza anterior. Aún cuando 
el tablero guarda las casillas, el estado de las piezas dejan de importar 
cuando una nueva pieza es ingresada al tablero por 
lo que el objeto pieza es eventualmente eliminado por el recolector de basura. Es 
en el objeto tablero donde las dimensiones del juego son almacenadas así como 
la puntuación\footnote{Se toma como puntuación al número de filas removidas 
durante una partida de Tetris.} del juego. Si bien la clase \texttt{Tablero} tiene 
definidos métodos que modifican el estado de una partida de Tetris, 
la lógica ha sido delegada a la clase \texttt{Tetris} que es la encargada de 
mandar a llamar a los métodos para modificar el juego.

La clase \texttt{Tetris} además de poseer un tablero, un historial de movimientos y un 
estado booleano llamado \texttt{\_game\_over} que indica si el juego ha finalizado, 
es la clase destinada a ser usada como interfaz para ser ejecutada por la heurística 
por lo que la lógica y datos del juego son modificados y obtenidos en su 
totalidad a través de sus métodos.

\begin{figure}[H]
\centering
\begin{tikzpicture}

\begin{umlpackage}[x=0 ,y=0]{tetris}

\umlclass[x=-6, y=6.3,scale=0.5]{Punto}{
- \_x : int \\
- \_y : int
}{
+ \_\_init\_\_(x : int, y : int) : Punto \\
+ get\_x() : int \\
+ get\_y() : int \\
+ set\_x(x : int) : void \\
+ set\_y(y : int) : void  \\
+ same(punto : Punto) : bool \\
+ clona() : Punto
}

\umlclass[x=-1, y=6.1,scale=0.5]{Casilla}{
- \_punto : Punto \\
- \_fija : bool \\
- \_tipo : Tipo
}{
+ \_\_init\_\_(punto : Punto, tipo : Tipo) : Casilla \\
+ get\_tipo() : Tipo \\
+ set\_tipo(tipo : Tipo) : void \\
+ get\_punto() : Punto \\
+ set\_punto(punto : Punto) : void  \\
+ set\_fija(fija : bool) : void \\
+ get\_fija() : bool \\
+ clona() : Casilla
}

\umlclass[x=4.5, y=5.3,scale=0.5]{Pieza}{
- \_tipo : Tipo \\
- \_orientacion : int \\
- \_posicion : Punto \\
- \_casillas : list<Casilla>
}{
+ \_\_init\_\_(tipo : Tipo, posicion : Punto) : Pieza \\
+ get\_orientacion() : int \\
+ set\_orientacion(orientacion : int) : void \\
+ set\_puntos(puntos : list<Casilla>) : void \\
+ get\_casillas\_self() : list<Casilla> \\
+ get\_puntos() : list<Punto> \\
+ rota() : void \\
+ mueve\_derecha() : void \\
+ mueve\_izquierda() : void \\
+ baja() : void \\
+ sube() : void \\
+ fija() : bool \\
+ casillas() : list<Casilla> \\
+ get\_tipo() : Tipo \\
+ clona() : Pieza \\
- \_\_get\_casillas() : list<Casilla>
}

\umlclass[x=-6, y=0.2,scale=0.5]{Tablero}{
- \_x : int \\
- \_y : int \\
- \_pieza : Pieza \\
- \_pieza\_anterior : Pieza \\
- \_limpia\_automatico : bool \\
- \_tablero : list<list<Casilla>> \\
- \_num\_tetris : int
}{
+ \_\_init\_\_(x : int, y : int) : Tablero \\
+ get\_casilla(x : int, y : int) : Casilla \\
+ set\_limpieza\_automatica(limpieza=True : bool) : void \\
+ get\_limpieza() : bool \\
+ copia\_tablero(tablero : list<list<Casilla>>) : void \\
+ asigna\_pieza\_anterior(pieza : Pieza) : void \\
+ asigna\_pieza\_clonada(pieza : Pieza) : void \\
+ punto\_inicial() : Punto \\
+ set\_pieza(tipo : Tipo) : bool \\
+ requiere\_pieza() : bool \\
+ altura\_maxima() : int \\
+ altura\_minima() : int \\
+ movimiento\_valido(move : Movimiento) : bool \\
+ juega\_movimiento(move) : bool \\
+ restaura\_pieza() : void \\
+ juega\_movimiento\_inverso(move : Movimiento) : bool \\
+ limpia() : void \\
+ puede\_limpiar() : bool $ \cup $ int \\
+ puede\_fijar() : bool \\
+ fijar() : bool $\cup$ void \\
+ print() : void \\
+ cuenta\_espacios(fila : int) : int \\
+ cuenta\_atrapados() : int \\
+ cuenta\_cubiertos() : int \\
+ num\_tetris() : int \\
+ set\_num\_tetris(num : int) : void \\
+ clona() : Tablero \\
- \_cuenta\_cubiertos(x : int, y : int) : int \\
- \_cuenta\_filas\_removidas() : void \\
- \_\_check\_cae() : bool \\
- \_\_check\_der() : bool \\
- \_\_check\_izq() : bool \\
- \_\_check\_gir() : bool \\
- \_\_actualiza\_pieza(puntos\_viejos : list<Punto>) : void \\
- \_\_revisa\_filas() : void \\
- \_\_limpia(fil : int) : void 
}

\umlclass[x=-.75, y=0.2,scale=0.5]{Tetris}{
- \_x : int \\
- \_y : int \\
- \_historial : list<Movimiento> \\
- \_tablero : Tablero \\
- \_piezas\_jugadas : int \\
- \_game\_over : bool \\
+ id : int
}{
+ \_\_init\_\_(x : int, y : int, tablero=None : Tablero) : Tetris \\
+ desactiva\_limpieza\_automatica() : void \\
+ activa\_limpieza\_automatica() : void \\
+ get\_altura() : int \\
+ get\_ancho() : int \\
+ set\_game\_over(flag : bool) : void \\ 
+ game\_over() : bool \\
+ set\_piezas\_jugadas(number : int) : void \\
+ altura\_maxima() : int \\
+ altura\_minima() : int \\
+ ultimo\_movimiento() : Movimiento \\
+ clona() : Tetris \\
+ set\_pieza(tipo=None : Tipo) : bool \\ 
+ puede\_fijar() : bool \\
+ mueve(move : Movimiento) : void \\
+ fija() : void \\
+ mueve\_o\_fija(move=None : Movimiento, restaura : bool) : bool \\
+ siguiente\_random(tipo=None : Tipo, move=None : Movimiento) : bool \\
+ limpia() : void \\
+ puede\_limpiar() : bool $ \cup $ int \\
+ get\_casilla(x : int, y : int) : Casilla \\
+ piezas\_jugadas() : int \\
+ set\_historial(historial : list<Movimiento>) : void \\
+ get\_historial() : list<Movimiento> \\
+ elimina\_historial(delta=1 : float) : None \\
+ num\_movimientos() : int \\
+ requiere\_pieza() : bool \\
+ imprime\_tablero() : void \\
+ movimiento\_valido(move : Movimiento) : bool \\
+ cuenta\_espacios(fila) : int \\
+ cuenta\_atrapados() : int \\
+ cuenta\_cubiertos() : int \\
+ num\_tetris() : int \\
+ \_\_eq\_\_(other : Object) : bool \\
+ \_\_ne\_\_(other : Object) : bool \\
+ \_\_hash\_\_() : int
}

\umlclass[x=5.7, y=-3,type=enum,scale=0.5]{Movimiento}{
DER, IZQ, CAE, GIR, FIJ
}{}

\umlclass[x=3.35, y=-3,type=enum,scale=0.5]{Tipo}{
I, RS, LG, T, RG, LS, Sq
}{}

\umlclass[x=4.5, y=0,scale=0.5]{tipo\_pieza}{}{
\umlstatic{+ instancia\_i(pos : Punto) : list<Casilla>} \\
\umlstatic{+ instancia\_rs(pos : Punto) : list<Casilla>} \\
\umlstatic{+ instancia\_lg(pos : Punto) : list<Casilla>} \\
\umlstatic{+ instancia\_t(pos : Punto) : list<Casilla>} \\
\umlstatic{+ instancia\_rg(pos : Punto) : list<Casilla>} \\
\umlstatic{+ instancia\_ls(pos : Punto) : list<Casilla>} \\
\umlstatic{+ instancia\_sq(pos : Punto) : list<Casilla>} \\
\umlstatic{+ get\_casillas(tipo : Tipo, posicion : Punto) : list<Casilla>} \\
\umlstatic{+ rota\_i(casillas : list<Casilla>) : list<Punto>} \\
\umlstatic{+ rota\_rs(casillas : list<Casilla>) : list<Punto>} \\
\umlstatic{+ rota\_lg(casillas : list<Casilla>) : list<Punto>} \\
\umlstatic{+ rota\_t(casillas : list<Casilla>) : list<Punto>} \\
\umlstatic{+ rota\_rg(casillas : list<Casilla>) : list<Punto>} \\
\umlstatic{+ rota\_ls(casillas : list<Casilla>) : list<Punto>} \\
\umlstatic{+ rota\_sq(casillas : list<Casilla>) : list<Punto>} \\
\umlstatic{+ rota(tipo : Tipo, casillas : list<Casilla>) : list<Punto>} 
}


\end{umlpackage}

%\umlimport[geometry=|-,anchor1=0, anchor2=west, name=import]{abc}{test}
%\umlimport[geometry=|-, name=import]{tetris}{test}
%\umlimport[geometry=|-,anchor1=north, anchor2=90, name=import]{test}{union}
\end{tikzpicture}
\caption{Diagramas UML del paquete \texttt{tetris}} \label{fig:uml-tetris}
\end{figure}

\subsection{Diseño de la heurística ABC}

Para un buen desempeño de la heurística y que funcione de acuerdo a la 
descripción de Karaboga, una colonia de abejas artificiales constituyen una 
colmena válida cuando existen tres tipos de abejas: las abejas exploradoras, 
las abejas empleadas o trabajadoras y las abejas observadoras. El tipo de 
cada abeja es independiente y puede ser representado por una constante, es por ello 
que se decidió realizar una enumeración con los tipos de abejas en 
la clase \texttt{Tipo\_Abeja}.

La clase, \texttt{Abeja} si bien sencilla y relativamente corta, está construida 
pensando en que el problema que la heurística resuelva no sea exclusivamente el 
juego de Tetris sino algo más general. Las abejas tienen un tipo, 
una fuente que es el punto de partida de cada abeja y es \texttt{None} al 
principio de la creación de cada objeto, una variable \texttt{\_limite} 
que indica el número de veces que una abeja puede visitar a una fuente, una 
variable \texttt{\_delta} que le sirve a las abejas observadoras para saber cuánto 
``alejarse'' de la fuente original.

Las abejas tienen además cuatro funciones que será explicada su implementación 
en el siguiente capítulo debido a la particularidad de que 
su definición tiene un impacto directo sobre los resultados. 
Cada función será usada dependiendo el rol 
que cada abeja tenga que cumplir en la colmena:

\begin{itemize}
\item \texttt{\_busca\_fuente:} Dado un estado inicial, esta función deberá 
manejar de alguna forma el cómo encontrar fuentes. La función debe recibir un objeto 
del tipo de la fuente y debe regresar un objeto de tipo de la fuente.

\item \texttt{\_observadoras:} Dado una fuente y un delta, esta función debe 
buscar fuentes vecinas partiendo de la fuente que reciba y ``alejándose'' una 
delta distancia. La función debe regresar un objeto de tipo de la fuente.

\item \texttt{\_nectar:} Es la función que habrá que definir con mayor cuidado 
puesto que todas las soluciones y comportamiento de la colmena se verá afectado 
por esta función. La función recibe una fuente y de alguna forma la evalúa para 
regresar un número. La forma en la que se trabaja en la colmena es \textit{higher 
is better}, esto quiere decir que entre más grande el número que esta función 
regrese, mejor la fuente es.

\item \texttt{\_explotar:} Esta es una función que recibe una fuente en un tiempo 
$T_{i}$ y regresa la evaluación de una nueva fuente en algún estado $T_{i+k}$. La nueva fuente 
sustituirá a la fuente anterior dentro de la colmena.
\end{itemize}

La colmena es el objeto que comunica a todas las abejas y fuentes, es el origen  
principal de datos de la heurística. La colmena tiene un tamaño constituido de 
la siguiente manera: 

\begin{displaymath}
  \frac{|colmena|}{2} = |Empleadas \cup Exploradoras| = |Observadoras|.
\end{displaymath}

Otro atributo importante de la colmena es la fuente inicial. Sin una fuente 
inicial, una abeja no podría empezar a explorar, trabajar ni observar debido a 
que debe existir un estado inicial para que todas las abejas trabajen en la colmena. 

\begin{figure}[H]
\centering
\begin{tikzpicture}

\begin{umlpackage}[x=0 ,y=0]{abc}

\umlclass[x=-4, y=1,scale=0.5]{Abeja}{
- \_tipo : Tipo\_Abeja \\
- \_id : int \\
- \_fuente : <T> \\
- \_limite : int \\
- \_delta : float \\
- \_busca\_fuente : $T \rightarrow T$ \\
- \_observadoras: $T \times float \rightarrow T$ \\
- \_nectar: $T \rightarrow float$ \\
- \_explotar: $T \rightarrow T$
}{
+ \_\_init\_\_(tipo : Tipo\_Abeja, id : int, delta : float) : Abeja \\
+ asigna\_funciones(**kwargs : list<function>) : void \\
+ get\_tipo() : Tipo\_Abeja \\
+ set\_tipo(tipo : Tipo\_Abeja) : void \\
+ get\_id() : int \\
+ set\_fuente(fuente : T) : void \\
+ get\_fuente() : T \\
+ observa\_solucion() : T \\
+ explota\_fuente() : T \\ 
+ busca\_fuente() : T \\
+ get\_nectar() : floar \\
+ get\_limite() : int \\
+ incrementa\_iteracion() : void \\
+ \_\_eq\_\_(other : Object) : bool \\
+ \_\_ne\_\_(other : Object) : bool \\
+ \_\_hash\_\_() : int
}

\umlclass[x=3, y=0,scale=0.5]{Colmena}{
- \_size : int \\
- \_fuente\_ini : <T> \\
- \_abejas : dict<Abeja> \\
- \_exploradoras : dict<Abeja> \\
- \_observadoras : dict<Abeja> \\
- \_empleadas : dict<Abeja> \\
- \_fuente\_abeja : dict<int> \\
- \_limite : int \\
- \_fuentes : dict<T> \\
- \_suma\_fuentes : float \\
- \_delta\_observacion : float \\
- \_busca\_fuente : $T \rightarrow T$ \\
- \_observadoras\_fun : $T \times float \rightarrow T$ \\
- \_nectar\_fun : $T \rightarrow float$ \\
- \_termino\_iteracion : $T \rightarrow V$ \\
- \_explotar\_fuente\_fun : $T \rightarrow T$ \\
- \_funcion\_comparativa : $T \rightarrow float$
}{
+ \_\_init\_\_(size : int, fuente\_inicial : T, limite : int, delta\_obs : float) : Colmena \\
+ set\_funcion\_nectar(calcular\_nectar\_function : $T \rightarrow float$) : void \\
+ set\_funcion\_buscar\_fuente(busca\_fuente\_function : $T \rightarrow T$) : void \\
+ set\_funcion\_observacion(determina\_observacion\_function : $T \times float \rightarrow T$) : void \\
+ set\_funcion\_explotar\_fuente(explorar\_fuente : $T \rightarrow T$) : void \\
+ set\_funcion\_comparativa(funcion\_comparativa : $T \rightarrow float$) :void \\
+ set\_funcion\_termino\_iteracion(termino\_iteracion : $T \rightarrow V$) : void \\
+ actualiza\_fuente\_inicial(fuente\_inicial : T) : void \\
+ inizializa\_abejas() : void \\
+ get\_nectar\_actual() : float \\
+ itera\_colmena() : void \\
+ get\_solucion\_final() : T \\
+ get\_mejor\_solucion() : T \\
- \_actualiza\_fuentes(fuente\_delta : T, fuente\_original : T) : void \\
- \_rueda\_ruleta(nectar : float, factor : float) : bool \\
- \_waggle\_dances() : void \\
- \_asigna\_funciones(id : int) : void \\
- \_itera\_exploradoras() : void \\
- \_itera\_empleadas() : void \\
- \_itera\_observadoras() : void \\
- \_llamada\_post\_iteracion() void
}

\umlclass[x=-4, y=-3.0,type=enum,scale=0.5]{Tipo\_Abeja}{
EMP, EXP, OBS
}{}


\end{umlpackage}

\end{tikzpicture}
\caption{Diagramas UML del paquete \texttt{abc}.} \label{fig:uml-abc}
\end{figure}

Durante una iteración de la colmena, las funciones principales a ser llamadas 
en orden son \texttt{\_itera\_exploradoras()}, \texttt{\_itera\_empleadas()} e 
\texttt{\_itera\_observadoras()}. Si a la colmena le fue asignada una 
función positerativa, al terminar una iteración a cada fuente $i$ se 
mandará a ejecutar la función \texttt{\_termino\_iteracion($i$)}.

Cada una de las funciones con prefijo \texttt{\_itera\_} se recorre al 
conjunto de abejas de cierto tipo y con ayuda de métodos y funciones 
auxiliares, realizan las acciones que cada abeja tiene destinada para que 
en la suma del conjunto de las acciones de todas las abejas contribuyan 
a seleccionar las mejores fuentes y construir una mejor solución en cada 
iteración.

\section{Comunicación heurística-emulador}

El diseño actual delega la responsabilidad del buen funcionamiento 
casi totalmente al objeto que hace uso de las clases 
\texttt{Colmena} y \texttt{Tetris}. Dichos objetos deberán invocar a 
algunas funciones previas a la interacción de ambos, como la 
asignación de las funciones a la colmena o desactivar la opción de 
quitar filas en automático dentro de un juego de Tetris. La clase 
encargada de realizar estas llamadas y generar un comportamiento 
armonioso entre ambos objetos es la clase \texttt{Abejas\_Tetris}. 


\begin{figure}[H]
\centering
\begin{tikzpicture}
\umlclass{Abejas\_Tetris}{
+ pierde : bool \\
+ online : bool \\
+ colmena : Colmena \\
+ lista\_piezas : list<Tipo> \\
- \_x : int \\
- \_y : int \\
- \_tetris : Tetris 
}{
+ \_\_init\_\_(online : bool, size\_colmena : int, limite\_it : int \\ delta: float, alto : int, ancho : int) : Abejas\_Tetris \\ 
+ set\_lista\_piezas(piezas : list<str>) : void \\
+ init\_colmena() : void \\
+ juega\_online(iteraciones : int, limpieza : bool) : Tetris \\
+ interactivo() : void \\
+ pinta\_historia(historia : list<Movimiento>) : void \\
$\qquad\qquad\qquad\qquad\qquad\qquad\dots$ 
}
\end{tikzpicture}
\caption{Atributos y métodos de la clase \texttt{Abejas\_Tetris}.} \label{fig:uml-a-t}
\end{figure}

Dentro de la clase \texttt{Abejas\_Tetris} se almacenan como atributos 
una partida de Tetris y una colmena con abejas, una bandera que nos dice 
el modo de juego, la lista de piezas que se jugarán y una segunda bandera 
que nos dice si el juego ha finalizado. Es necesario definir también las 
funciones de evaluación que las abejas 
usarán ya que es en este punto donde la colmena debe recibir cada una 
de las funciones para la evaluación de los tableros que más adelante se 
discutirán.

\section{Visualización de datos}

Una vez que se haya creado un historial de ejecuciones, es natural querer 
ver de manera secuencial las decisiones tomadas por la heurística y seguir paso 
a paso cada iteración del tablero para analizar el impacto de la función de costo 
y el comportamiento de la colmena de una manera visual. Dado que se necesita tanto de la solución 
propuesta por la colmena y las operaciones sobre un tablero (que tiene de estado 
inicial una partida nueva), el lugar más conveniente para crear las funciones 
de visualización es el mismo lugar en el que ocurre la lógica de ambos.

El método \texttt{pinta\_solucion()} dentro de la clase \texttt{Abejas\_Tetris}, 
recibe una lista de movimientos e inicializa los parámetros de la biblioteca 
de visualización \textit{pygame}. Mientras la lista que recibe no esté vacía o el 
juego no llegue a un estado donde el usuario\footnote{El usuario en este 
caso particular es la colmena.} pierda, el método 
mandará a la lógica del juego la orden de moverse a un estado siguiente al ejecutar 
el movimiento que saque de la cabeza de la lista, posteriormente mandará a 
llamar el método \texttt{dibuja()} que tomará el estado del tablero y las fichas 
para actualizar su posición y dibujar una nueva versión del juego.

\begin{figure}[H]
\centering
\begin{tikzpicture}
\umlclass{Abejas\_Tetris}{
$
\qquad\qquad\qquad\qquad\qquad\quad
\dots
$
}{
$\qquad\qquad\qquad\qquad\qquad\quad\dots$ \\
+ pinta\_solucion(solucion : list<Movimiento>) : void \\
+ set\_gui() : void \\
+ dibuja() : void \\
- \_\_dibuja\_fichas() : void \\
- \_\_dibuja\_tablero() : void
}
\end{tikzpicture}
\caption{Funciones y métodos de visualización.} \label{fig:uml-vis}
\end{figure}

Existen tres maneras de visualizar una partida y estas dependerán del modo 
de juego: cuando se ha terminado de realizar una ejecución de la heurística, ya sea por 
medio de una búsqueda de semillas o una partida aleatoria, el modo interactivo 
que espera a que las indicaciones sean introducidas desde la terminal. El último modo 
es una forma de visualizar resultados anteriores y que sólo necesita una lista 
de movimientos para dibujar una partida de Tetris.


\section{Funciones y métodos adicionales}

Como parte del análisis del problema se realizaron decisiones de diseño 
que pueden ser importantes para entender el funcionamiento del sistema. 
Algunas de estas decisiones fueron el resultado de problemas que se encontraron 
en diseños previos como parte de la optimización o manejo funciones intrínsecas 
del lenguaje.

Algunas de las decisiones tomadas parecieran no son congruentes con 
características del lenguaje o los patrones de diseño más usados, sino 
que responden a la experiencia de una primera implementación en otro 
lenguaje\footnote{Dicho proyecto fue creado en el lenguaje C y se puede ver 
el funcionamiento y código fuente en la siguiente dirección: 
\url{https://github.com/ricardorodab/abejas-tetris}.}.

\subsection{Creación y rotaciones de las piezas}

La creación de las casillas de las piezas no es generado dentro de la clase 
\texttt{Pieza}, sino dentro del archivo \texttt{./abejas\_tetris/tetris/tipo\_pieza.py} 
debido a que con sólo conocer el punto donde se desea colocar el objeto y su orientación, 
es posible deducir siempre con un número constante de operaciones (a lo más tres) la 
posición del resto de las casillas. 

La única operación que realizan las casillas como parte del objeto pieza, la 
operación de rotación, 
tampoco se encuentra dentro de la clase sino que se delega a la función 
\texttt{rota()} dentro del mismo archivo donde se crean. La decisión detrás de 
separar estas funciones de la clase \texttt{Pieza} es que no se necesita información adicional 
ni operar con los atributos del objeto para generar la acción de crear y 
girar las casillas, generando una mayor limpieza en el código y aumentando una 
velocidad mayor al no depender del estado de las casillas previas de una pieza al 
girar sino de operaciones constantes.

\subsection{Archivo de parámetros globales}

Dentro del archivo \texttt{./etc/config.cfg} existen variables globales que se 
usan para la ejecución del proyecto y dentro de éste se pueden colocar 
parámetros de la colmena como su tamaño, las deltas de las abejas observadoras, 
el límite de veces que una abeja trabaja sobre una fuente específica así como 
parámetros del juego de Tetris como el alto y ancho del tablero.

Todos estos valores son de vital importancia durante la ejecución de 
la heurística y el menor cambio de éstos, produce una salida de diferente calidad.

Para poder instanciar un objeto de algunas clases del proyecto 
es necesario haber obtenido los parámetros que se encuentran dentro 
del archivo con la función \texttt{get\_config()} que se encuentra en la ruta  
\texttt{./abejas\_tetris/parse\_config.py}. 

La creación de un archivo de parámetros independientes responde a la necesidad 
de experimentar con los distintos pesos que se les da a los valores de 
ejecución del programa.


\subsection{Generador de números aleatorios}

Debido a la naturaleza de las heurísticas y el proyecto, en muchas partes partes 
del código se pueden observar que se utilizan funciones como \texttt{get\_random()}, 
\texttt{get\_randrange()} y \texttt{get\_randbits()}. Debido a que durante la experimentación 
es deseable poder mantener un control sobre los resultados, se usan lo 
que se conocen como \textit{semillas generadoras de números pseudoaletorios} para 
asegurar que el programa sea replicable. 
Para forzar sólo un objeto que genere 
los números aleatorios, se usa el patrón de diseño conocido como
\textit{singleton}, para solicitar sólo un generador que funcione a la vez con sólo 
una semilla.

Para colocar una semilla se usa una función llamada \texttt{set\_random()} que 
recibe como parámetro la semilla como un número entero. 
Junto este método se escribió un algoritmo que tiene como propósito el 
hacer búsqueda de semillas generadoras con resultados favorables 
que obtengan buenas partidas de Tetris. Para realizar la búsqueda se debe colocar 
la constante \texttt{-1} en la variable \texttt{SEMILLA} dentro del archivo 
de parámetros globales, colocar el número de semillas que se desean probar 
en la variable \texttt{SEMILLA\_ITERA} y correr el \textit{script} 
\texttt{./scripts/llena-semillero.sh} que se encarga de crear el archivo con las 
semillas aleatorias.

\subsection{Constantes}

Existen para este sistema valores constantes que no se desean cambiar como 
son el color de los tipos de las distintas piezas, el tamaño de los bloques o 
el tamaño de los márgenes. Además de ser un buen diseño y agregarle legibilidad al 
código, se tomó la decisión de crear un archivo autónomo ya que estos valores son 
completamente independientes de la ejecución de la heurística y del juego. Los 
valores constantes dentro del proyecto se usan para la visualización de los 
resultados y son propios de la biblioteca \textit{pygame}.

Si por alguna razón se desearan cambiar los valores constantes, todos se mantienen 
en la ruta \texttt{./abejas\_tetris/constantes.py}. La constante \texttt{ESCALADO}, 
es la responsable del tamaño de los objetos de visualización por lo que si se modificara, 
la interfaz gráfica se vería afectada haciendo que su visualización sea de menor o mayor tamaño.


%%% Local Variables:
%%% mode: latex
%%% TeX-master: "../main"
%%% End:
