\chapter{Conclusiones y trabajo futuro}

El origen y motivación de realizar este trabajo surgió de una primera 
implementación realizada en un curso optativo que lleva por nombre 
\textit{Seminario de Ciencias de la Computación B} y tenía por tema principal 
\textit{Heurísticas de Optimización Combinatoria}. 
Si bien la primera implementación (que se puede encontrar 
en la siguiente dirección: \url{https://github.com/ricardorodab/abejas-tetris}) 
conseguía realizar combinaciones de juego con cierto grado de éxito, 
careció (por una cuestión de tiempo) del detalle de diseño, desarrollo de funciones 
de costo y pruebas que sí son realizadas en este trabajo.

Esta segunda implementación se originó como un experimento para entender el comportamiento 
de una heurística documentada a un problema ampliamente conocido. 
Es el deseo del autor que la compilación de terminología ocupada en conjunto a 
las definiciones mencionadas, teoremas, ejemplos y experimentos resulten 
de una mayor facilidad de entendimiento al lector para despertar su curiosidad 
y si lo desea, continuar las investigaciones que este trabajo introduce.

Se tomaron las bases teóricas para explicar por qué la selección del problema 
de Tetris. Se explicó el funcionamiento de las heurísticas como 
lo es la \textit{Colonia de Abejas Artificiales} o 
\textit{ABC} por sus siglas en inglés. Se desarrolló 
la justificación de tomar una heurística sobre otros métodos resolutivos y se argumentó  
el porqué del diseño que se implementó para realizar la experimentación.

Aunque el enfoque dado a la forma de resolver el problema de Tetris fue 
reducido a dos funciones, es parte del diseño la facilidad de comunicarle 
a la heurística ABC la implementación de nuevas 
funciones de desempeño para extender la experimentación a nuevos métodos. 

El conjunto de resultados generados por la implementación muestran que: 

\begin{enumerate}

\item Sin importar que el problema de Tetris sea un problema \textsl{NP}-completo, 
existen métodos prácticos que pueden encontrar soluciones de buena calidad.

\item La heurística ABC es un buen método de resolución de problemas ya  
que la función de costo de la que depende sus resultados es independiente al 
comportamiento del colectivo en la colmena.

\end{enumerate}

Un paso siguiente a la solución propuesta en este trabajo es crear 
evaluaciones \textit{offline} tomando en consideración las piezas siguientes 
como origen de las fuentes de la heurística. Esta propuesta queda fuera del 
alcance de este trabajo debido a que las funciones y el diseño usado deberán 
ser modificados de manera profunda; sin embargo, la heurística debería 
mantener siempre la misma estructura sin cambios considerables. 

Otra optimización necesaria sobre el diseño de la heurística es la 
paralelización de las tareas dentro de la colmena. Cada tipo de abeja 
puede ser dividida a un proceso productor-consumidor independiente que 
espere a ser alimentada de fuentes para su trabajo. El resultado 
mostrado no cambiará con dicha optimización pero el tiempo de respuesta 
por parte de la heurística se vería reducido.

Para finalizar este trabajo hay que mencionar que las metas propuestas 
como tiempo de juego y número de filas eliminadas, no sólo fueron alcanzadas 
sino que superadas por más del doble y se plantea la pregunta \textit{¿podría la 
heurística vencer a un jugador experto en tiempo de juego real con una mejor 
función de costo?}